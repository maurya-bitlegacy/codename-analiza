\mode<presentation>
{
  %\usetheme{umbc4}
  %\setbeamercovered{dynamic}
}
\usepackage[english]{babel}
% or whatever

\usepackage[latin1]{inputenc}
% or whatever

\usepackage{times}
\usepackage[T1]{fontenc}
% Or whatever. Note that the encoding and the font should match. If T1
% does not look nice, try deleting the line with the fontenc.
%--------------------------------------------------------------------------------------------
%--- Other packages
\usepackage{pstricks}
\usepackage{pst-node}
\usepackage{pst-rel-points}
\usepackage{xspace}

%%%%%% Declarations Defined in custom-foils
%\setfootline{\insertshortauthor, \insertshortinstitute \hfill   \insertshorttitle \hfill \insertframenumber/\inserttotalframenumber} 
\newcommand{\mypart}[1]{%
  \frame[plain]{\mbox{}\vfill \psshadowbox{\huge #1} \vfill\mbox{}}
}
\author[karkare]{Amey Karkare \\ \url{karkare@cse.iitk.ac.in}}
\date[]{\scalebox{0.3}{\includegraphics{iitklogo.epsi}}}%
\institute[CSE, IITK]{\url{http://www.cse.iitk.ac.in/~karkare/cs738}\\Department of CSE, IIT Kanpur}
\title[CS738]{CS738: Advanced Compiler Optimizations}
\subtitle[]{}


\subtitle[]{\ \\{\LARGE Flow Graph Theory}\\}

\newcommand{\bbent}{{\it Entry}\xspace}
\newcommand{\bbext}{{\it Exit}\xspace}

\newcommand{\IN}{{\sf IN}\xspace}
\newcommand{\OUT}{{\sf OUT}\xspace}

\newcommand{\Pred}{{\sf PRED}\xspace}
\newcommand{\Succ}{{\sf SUCC}\xspace}

\newcommand{\Kill}{{\sf KILL}\xspace}
\newcommand{\Gen}{{\sf GEN}\xspace}

\newcommand{\cmt}[1]{{}}
\newcommand{\meet}{\ensuremath{\bigwedge}\xspace}
\newcommand{\join}{\ensuremath{\bigvee}\xspace}
\newcommand{\glb}{{\sf glb}\xspace}
\newcommand{\lub}{{\sf lub}\xspace}

%% Framed minipage %%%%%%%%%%%%%%%%%%%%%%%%%%%%%%%%%%%%
\newsavebox\TestBox
\newenvironment{frmminipage}[2][]
{\begin{lrbox}{\TestBox}\begin{minipage}[#1]{#2}}
{\end{minipage}\end{lrbox}\fbox{\usebox{\TestBox}}}


\newcommand{\mylabbox}[2]{{\small #1}\psframebox{#2}\phantom{\small #1}}

\newcommand{\tossa}{%
  \ensuremath{\stackrel{SSA\quad\ }{\scalebox{3}{$\Rightarrow$}}}}

\newcommand{\jstfy}{\mbox{\phantom{XX}}}

\newcommand{\dom}{\operatorname{dom}}
\newcommand{\sdom}{\operatorname{sdom}}
\newcommand{\idom}{\operatorname{idom}}

\newcommand{\DOM}{\operatorname{DOM}}
\newcommand{\DF}{\operatorname{DF}}

\begin{document}

\frame{\titlepage}
\frame{
  \frametitle{Agenda}
  \begin{itemize}
  \item Speeding up DFA
  \item Depth of a flow graph
  \item Natural Loops
  \end{itemize}
}

\frame{
  \frametitle{Acknowledgement}
  Rest of the slides based on the material at  \url{http://infolab.stanford.edu/~ullman/dragon/w06/} \url{w06.html}
}


\frame{
  \frametitle{Speeding up DFA}
  \begin{itemize}[<+->]
  \item Proper ordering of nodes of a flow graph speeds up the
    iterative algorithms: {\bf depth-first ordering}.
  \item ``Normal'' flow graphs have a surprising property ---
    {\bf reducibility} --- that simplifies several matters.
  \item Outcome: few iterations ``normally'' needed.
  \end{itemize}
}

\frame{
  \frametitle{Depth-First Search}
  \begin{itemize}[<+->]
  \item Start at entry.
  \item If you can follow an edge to an unvisited node, do so.
  \item If not, backtrack to your {\em parent} (node from which
    you were visited).
  \end{itemize}
}

\frame{
  \frametitle{Depth-First Spanning Tree (DFST)}
  \begin{itemize}[<+->]
  \item Root = \bbent.
  \item Tree edges are the edges along which we first visit the
    node at the head.
  \end{itemize}
}

\frame{
  \frametitle{DFST Example}
  \begin{center}
    \psset{unit=1mm}
    \begin{pspicture}(0,0)(30,40)
      \putnode{n1}{origin}{15}{35}{\pscirclebox{1}}
      \putnode{n2}{n1}{-15}{-15}{\pscirclebox{2}}
      \putnode{n3}{n1}{0}{-30}{\pscirclebox{3}}
      \putnode{n4}{n2}{30}{0}{\pscirclebox{4}}
      \putnode{n5}{n3}{30}{0}{\pscirclebox{5}}

      \ncline[nodesep=-0.9]{->}{n1}{n2}
      \ncline[nodesep=-0.9]{->}{n1}{n4}
      \ncline[nodesep=-0.9]{->}{n2}{n3}
      \ncline[nodesep=-0.9]{->}{n4}{n3}
      \ncline[nodesep=-0.9]{->}{n4}{n5}

      \nccurve[nodesep=-0.3,angleA=-180,angleB=-180,ncurv=2]{->}{n3}{n1}
      \nccurve[nodesep=-0.7,angleA=45,angleB=45]{->}{n5}{n4}

      \pause
      \ncline[linecolor=red, linewidth=1.2, offset=1, nodesep=-0.9]{-}{n1}{n2}
      \pause
      \ncline[linecolor=red, linewidth=1.2, offset=1, nodesep=-0.9]{-}{n2}{n3}
      \pause
      \ncline[linecolor=red, linewidth=1.2, offset=1, nodesep=-0.9]{-}{n1}{n4}
      \pause
      \ncline[linecolor=red, linewidth=1.2, offset=1, nodesep=-0.9]{-}{n4}{n5}

      
    \end{pspicture}
  \end{center}
}

\frame{
  \frametitle{Depth-First Node Order}
  \begin{itemize}[<+->]
  \item The reverse of the order in which a DFS {\bf retreats} from the nodes.
  \item Alternatively, reverse of postorder traversal of the tree.
  \end{itemize}
}

\frame{
  \frametitle{DF Order Example}
  \begin{center}
    \psset{unit=1mm}
    \begin{pspicture}(0,0)(30,40)
      \putnode{n1}{origin}{15}{35}{\pscirclebox{1}}
      \putnode{n2}{n1}{-15}{-15}{\pscirclebox{4}}
      \putnode{n3}{n1}{0}{-30}{\pscirclebox{5}}
      \putnode{n4}{n2}{30}{0}{\pscirclebox{2}}
      \putnode{n5}{n3}{30}{0}{\pscirclebox{3}}

      \ncline[nodesep=-0.9]{->}{n1}{n2}
      \ncline[nodesep=-0.9]{->}{n1}{n4}
      \ncline[nodesep=-0.9]{->}{n2}{n3}
      \ncline[nodesep=-0.9]{->}{n4}{n3}
      \ncline[nodesep=-0.9]{->}{n4}{n5}

      \nccurve[nodesep=-0.3,angleA=-180,angleB=-180,ncurv=2]{->}{n3}{n1}
      \nccurve[nodesep=-0.7,angleA=45,angleB=45]{->}{n5}{n4}

      \ncline[linecolor=red, linewidth=1.2, offset=1, nodesep=-0.9]{-}{n1}{n2}
      \ncline[linecolor=red, linewidth=1.2, offset=1, nodesep=-0.9]{-}{n2}{n3}
      \ncline[linecolor=red, linewidth=1.2, offset=1, nodesep=-0.9]{-}{n1}{n4}
      \ncline[linecolor=red, linewidth=1.2, offset=1, nodesep=-0.9]{-}{n4}{n5}

      
    \end{pspicture}
  \end{center}
}

\frame{
  \frametitle{Four Kind of Edges}
  \begin{enumerate}[<+->]
  \item Tree edges.
  \item {\bf Forward edges}: node to proper descendant.
  \item {\bf Retreating edges}: node to ancestor.
  \item {\bf Cross edges}: between two node, neither of which is
    an ancestor of the other.
  \end{enumerate}
}

\frame{
  \frametitle{A Little Magic}
  \begin{itemize}[<+->]
  \item Of these edges, only retreating edges go from high to low in DF order.
  \item Most surprising: all cross edges go right to left in the DFST.
    \begin{itemize}
    \item Assuming we add children of any node from the left.
    \end{itemize}
  \end{itemize}
}

\frame{
  \frametitle{Example: Non-Tree Edges}
  \begin{center}
    \psset{unit=1mm}
    \begin{pspicture}(0,0)(30,40)
      \putnode{n1}{origin}{15}{35}{\pscirclebox{1}}
      \putnode{n2}{n1}{-15}{-15}{\pscirclebox{4}}
      \putnode{n3}{n1}{0}{-30}{\pscirclebox{5}}
      \putnode{n4}{n2}{30}{0}{\pscirclebox{2}}
      \putnode{n5}{n3}{30}{0}{\pscirclebox{3}}

      \ncline[nodesep=-0.9]{->}{n1}{n2}
      \ncline[nodesep=-0.9]{->}{n1}{n4}
      \ncline[nodesep=-0.9]{->}{n2}{n3}
      \ncline[nodesep=-0.9]{->}{n4}{n3}
      \ncline[nodesep=-0.9]{->}{n4}{n5}

      \nccurve[nodesep=-0.3,angleA=-180,angleB=-180,ncurv=2]{->}{n3}{n1}
      \nccurve[nodesep=-0.7,angleA=45,angleB=45]{->}{n5}{n4}
      \nccurve[nodesepA=-0.7,angleA=45,angleB=0]{->}{n1}{n5}

      \ncline[linecolor=red, linewidth=1.2, offset=1, nodesep=-0.9]{-}{n1}{n2}
      \ncline[linecolor=red, linewidth=1.2, offset=1, nodesep=-0.9]{-}{n2}{n3}
      \ncline[linecolor=red, linewidth=1.2, offset=1, nodesep=-0.9]{-}{n1}{n4}
      \ncline[linecolor=red, linewidth=1.2, offset=1, nodesep=-0.9]{-}{n4}{n5}

      \pause
      \putnode{re}{n1}{0}{10}{Retreating}
      \putnode{tre1}{re}{-11}{-10}{}
      \ncline[linestyle=dashed]{->}{re}{tre1}
      \putnode{tre2}{re}{20}{-21}{}
      \ncline[linestyle=dashed]{->}{re}{tre2}

      \pause
      \putnode{fw}{n1}{30}{5}{Forward}
      \putnode{tfw}{fw}{-10}{-8}{}
      \ncline[linestyle=dashed]{->}{fw}{tfw}
      
      \pause
      \putnode{ce}{n1}{10}{-40}{Cross}
      \putnode{tce}{ce}{0}{19}{}
      \ncline[linestyle=dashed]{->}{ce}{tce}
      
    \end{pspicture}
  \end{center}
}

\frame{
  \frametitle{Roadmap}
  \begin{itemize}[<+->]
  \item ``Normal'' flow graphs are ``{\bf reducible}.''
  \item ``{\bf Dominators}'' needed to explain reducibility.
  \item In reducible flow graphs, loops are well defined,
    retreating edges are unique (and called ``{\bf back}''
    edges).
  \item Leads to relationship between DF order and efficient
    iterative algorithm.
  \end{itemize}
}

\frame{
  \frametitle{Dominators}
  \begin{itemize}[<+->]
  \item Node $d$ {\bf dominates} node $n$ if every path from the
    \bbent to $n$ goes through $d$.
  \item \mbox{[Exercise]} A forward-intersection iterative algorithm
    for finding dominators.
  \item Quick observations:
    \begin{itemize}
    \item Every node dominates itself.
    \item The entry dominates every node.
    \end{itemize}
  \end{itemize}
}

\frame{
  \frametitle{Example: Dominators}
  \begin{center}
    \psset{unit=1mm}
    \begin{pspicture}(0,0)(30,40)
      \putnode{n1}{origin}{-10}{35}{\pscirclebox{1}}
      \putnode{n2}{n1}{-15}{-15}{\pscirclebox{4}}
      \putnode{n3}{n1}{0}{-30}{\pscirclebox{5}}
      \putnode{n4}{n2}{30}{0}{\pscirclebox{2}}
      \putnode{n5}{n3}{30}{0}{\pscirclebox{3}}

      \ncline[nodesep=-0.9]{->}{n1}{n2}
      \ncline[nodesep=-0.9]{->}{n1}{n4}
      \ncline[nodesep=-0.9]{->}{n2}{n3}
      \ncline[nodesep=-0.9]{->}{n4}{n3}
      \ncline[nodesep=-0.9]{->}{n4}{n5}

      \nccurve[nodesep=-0.3,angleA=-180,angleB=-180,ncurv=2]{->}{n3}{n1}
      \nccurve[nodesep=-0.7,angleA=45,angleB=45]{->}{n5}{n4}

      \ncline[linecolor=red, linewidth=1.2, offset=1, nodesep=-0.9]{-}{n1}{n2}
      \ncline[linecolor=red, linewidth=1.2, offset=1, nodesep=-0.9]{-}{n2}{n3}
      \ncline[linecolor=red, linewidth=1.2, offset=1, nodesep=-0.9]{-}{n1}{n4}
      \ncline[linecolor=red, linewidth=1.2, offset=1, nodesep=-0.9]{-}{n4}{n5}
      \putnode{tab}{origin}{40}{35}{
        \begin{tabular}{|c|c|} \hline
          {\bf Node} & {\bf Dominators} \\ \hline \hline
          1 & \onslide<2->{1} \\ \hline
          2 & \onslide<3->{1, 2} \\ \hline
          3 & \onslide<4->{1, 2, 3} \\ \hline
          4 & \onslide<5->{1, 4} \\ \hline
          5 & \onslide<6->{1, 5} \\ \hline
        \end{tabular}}
    \end{pspicture}
  \end{center}
}

\frame{
  \frametitle{Common Dominator Cases}
  \begin{itemize}[<+->]
  \item The test of a while loop dominates all blocks in the loop
    body.
  \item The test of an if-then-else dominates all blocks in
    either branch.
  \end{itemize}
}

\frame{
  \frametitle{Back Edges}
  \begin{itemize}[<+->]
  \item An edge is a {\bf back edge} if its head dominates its tail.
  \item {\bf Theorem:} Every back edge is a retreating edge in
    every DFST of every flow graph.
    \begin{itemize}
    \item Proof? Discuss/Exercise
    \item Converse almost always true, but not always.
    \end{itemize}
  \end{itemize}
}

\frame{
  \frametitle{Example: Back Edges}
  \begin{center}
    \psset{unit=1mm}
    \begin{pspicture}(0,0)(30,40)
      \putnode{n1}{origin}{-10}{35}{\pscirclebox{1}}
      \putnode{n2}{n1}{-15}{-15}{\pscirclebox{4}}
      \putnode{n3}{n1}{0}{-30}{\pscirclebox{5}}
      \putnode{n4}{n2}{30}{0}{\pscirclebox{2}}
      \putnode{n5}{n3}{30}{0}{\pscirclebox{3}}

      \ncline[nodesep=-0.9]{->}{n1}{n2}
      \ncline[nodesep=-0.9]{->}{n1}{n4}
      \ncline[nodesep=-0.9]{->}{n2}{n3}
      \ncline[nodesep=-0.9]{->}{n4}{n3}
      \ncline[nodesep=-0.9]{->}{n4}{n5}

      \nccurve[nodesep=-0.3,angleA=-180,angleB=-180,ncurv=2]{->}{n3}{n1}
      \nccurve[nodesep=-0.7,angleA=45,angleB=45]{->}{n5}{n4}

      \ncline[linecolor=red, linewidth=1.2, offset=1, nodesep=-0.9]{-}{n1}{n2}
      \ncline[linecolor=red, linewidth=1.2, offset=1, nodesep=-0.9]{-}{n2}{n3}
      \ncline[linecolor=red, linewidth=1.2, offset=1, nodesep=-0.9]{-}{n1}{n4}
      \ncline[linecolor=red, linewidth=1.2, offset=1, nodesep=-0.9]{-}{n4}{n5}
      \putnode{tab}{origin}{40}{35}{
        \begin{tabular}{|c|c|} \hline
          {\bf Node} & {\bf Dominators} \\ \hline \hline
          1 & {1} \\ \hline
          2 & {1, 2} \\ \hline
          3 & {1, 2, 3} \\ \hline
          4 & {1, 4} \\ \hline
          5 & {1, 5} \\ \hline
        \end{tabular}}\pause
      \nccurve[linecolor=blue,linewidth=1,nodesep=-0.3,angleA=-180,angleB=-180,ncurv=2]{->}{n3}{n1}
      \nccurve[linecolor=blue,linewidth=1,nodesep=-0.7,angleA=45,angleB=45]{->}{n5}{n4}

    \end{pspicture}
  \end{center}
}

\frame{
  \frametitle{Reducible Flow Graphs}
  \begin{itemize}[<+->]
  \item A flow graph is {\bf reducible} if every retreating edge
    in any DFST for that flow graph is a back edge.
  \item {\bf Testing reducibility:} Take any DFST for the flow
    graph, remove the back edges, and check that the result is
    acyclic.
  \end{itemize}
}

\frame{
  \frametitle{Example: Remove Back Edges}
  \begin{center}
    \psset{unit=1mm}
    \begin{pspicture}(0,0)(30,40)
      \putnode{n1}{origin}{-10}{35}{\pscirclebox{1}}
      \putnode{n2}{n1}{-15}{-15}{\pscirclebox{4}}
      \putnode{n3}{n1}{0}{-30}{\pscirclebox{5}}
      \putnode{n4}{n2}{30}{0}{\pscirclebox{2}}
      \putnode{n5}{n3}{30}{0}{\pscirclebox{3}}

      \ncline[nodesep=-0.9]{->}{n1}{n2}
      \ncline[nodesep=-0.9]{->}{n1}{n4}
      \ncline[nodesep=-0.9]{->}{n2}{n3}
      \ncline[nodesep=-0.9]{->}{n4}{n3}
      \ncline[nodesep=-0.9]{->}{n4}{n5}

      %\nccurve[nodesep=-0.3,angleA=-180,angleB=-180,ncurv=2]{->}{n3}{n1}
      %\nccurve[nodesep=-0.7,angleA=45,angleB=45]{->}{n5}{n4}

      \ncline[linecolor=red, linewidth=1.2, offset=1, nodesep=-0.9]{-}{n1}{n2}
      \ncline[linecolor=red, linewidth=1.2, offset=1, nodesep=-0.9]{-}{n2}{n3}
      \ncline[linecolor=red, linewidth=1.2, offset=1, nodesep=-0.9]{-}{n1}{n4}
      \ncline[linecolor=red, linewidth=1.2, offset=1, nodesep=-0.9]{-}{n4}{n5}
      \putnode{tab}{origin}{40}{35}{
        \begin{tabular}{|c|c|} \hline
          {\bf Node} & {\bf Dominators} \\ \hline \hline
          1 & {1} \\ \hline
          2 & {1, 2} \\ \hline
          3 & {1, 2, 3} \\ \hline
          4 & {1, 4} \\ \hline
          5 & {1, 5} \\ \hline
        \end{tabular}}
      \only<1>{\nccurve[linecolor=blue,linewidth=1,nodesep=-0.3,angleA=-180,angleB=-180,ncurv=2]{->}{n3}{n1}
        \nccurve[linecolor=blue,linewidth=1,nodesep=-0.7,angleA=45,angleB=45]{->}{n5}{n4}}
      \only<2->{\putnode{rem}{origin}{40}{0}{Remaining graph is acyclic.}}
    \end{pspicture}
  \end{center}
  
}

\frame{
  \frametitle{Why Reducibility?}
  \begin{itemize}
  \item {\bf Folk theorem:}  All flow graphs in practice are reducible.
  \item {\bf Fact:} If you use only while-loops, for-loops,
    repeat-loops, if-then(-else), break, and continue, then your
    flow graph {\bf is} reducible.
  \end{itemize}
}

\frame{
  \frametitle{Example: Nonreducible Graph}
  \begin{center}
    \psset{unit=1mm}
    \begin{pspicture}(0,0)(80,50)
      \putnode{a0}{origin}{40}{40}{%
        \pscirclebox[fillcolor=blue!20,fillstyle=solid]{A}}
      \putnode{b0}{a0}{-13}{-23}{%
        \pscirclebox[fillcolor=blue!20,fillstyle=solid]{B}}
      \putnode{c0}{a0}{13}{-23}{%
        \pscirclebox[fillcolor=blue!20,fillstyle=solid]{C}}
      \ncline[nodesep=-0.5]{->}{a0}{b0}
      \ncline[nodesep=-0.5]{->}{a0}{c0}
      \ncline{->}{b0}{c0}
      \nccurve[nodesep=-0.9,angleA=-135,angleB=-45]{->}{c0}{b0}
      \putnode{bc0}{b0}{15}{0}{}
      \putnode{bc1}{b0}{11}{-6}{}
      \putnode{l0}{bc0}{0}{-13}{}
      \putnode{l1}{l0}{-4}{0}{}
      \pause
      \putnode{lab}{b0}{13}{-20}{
        \begin{tabular}{p{45mm}}
          In any DFST, one of these edges will be a retreating edge.
      \end{tabular}}
      \ncline[linestyle=dotted,dotsep=0.3]{->}{l0}{bc0}
      \ncline[linestyle=dotted,dotsep=0.3]{->}{l1}{bc1}
      
      \pause
      \putnode{a1}{a0}{-30}{0}{%
        \pscirclebox[fillcolor=blue!20,fillstyle=solid]{A}}
      \putnode{b1}{a1}{0}{-13}{%
        \pscirclebox[fillcolor=blue!20,fillstyle=solid]{B}}
      \putnode{c1}{a1}{0}{-26}{%
        \pscirclebox[fillcolor=blue!20,fillstyle=solid]{C}}
      \ncline[nodesep=-0.5]{->}{a1}{b1}
      \nccurve[nodesep=-0.1]{->}{a1}{c1}
      \ncline{->}{b1}{c1}
      \nccurve[nodesep=-0.1,angleA=-195,angleB=-165,
        linecolor=red,linewidth=.3]{->}{c1}{b1}
      \pause
      \putnode{a2}{a0}{30}{0}{%
        \pscirclebox[fillcolor=blue!20,fillstyle=solid]{A}}
      \putnode{b2}{a2}{0}{-13}{%
        \pscirclebox[fillcolor=blue!20,fillstyle=solid]{C}}
      \putnode{c2}{a2}{0}{-26}{%
        \pscirclebox[fillcolor=blue!20,fillstyle=solid]{B}}
      \ncline[nodesep=-0.5]{->}{a2}{b2}
      \nccurve[nodesep=-0.1]{->}{a2}{c2}
      \ncline{->}{b2}{c2}
      \nccurve[nodesep=-0.1,angleA=-195,angleB=-165,
        linecolor=red,linewidth=.3]{->}{c2}{b2}
\end{pspicture}
  \end{center}
  
}

\frame{
  \frametitle{Why Care About Back/Retreating Edges?}
  \begin{itemize}[<+->]
  \item Proper ordering of nodes during iterative algorithm
    assures number of passes limited by the number of ``nested''
    back edges.
  \item Depth of nested loops upper-bounds the number of nested
    back edges.
  \end{itemize}
}

\frame{
  \frametitle{DF Order and Retreating Edges}
  \begin{itemize}[<+->]
  \item Suppose that for a RD analysis, we visit nodes during
    each iteration in DF order.
  \item The fact that a definition $d$ reaches a block will
    propagate in one pass along any increasing sequence of
    blocks.
  \item When $d$ arrives along a retreating edge, it is too late to
    propagate $d$ from \OUT to \IN.
  \end{itemize}
}

\frame{
  \frametitle{Example: DF Order}
  \begin{center}
    \psset{unit=1mm}
    \begin{pspicture}(-35,0)(60,60)
      %\psframe(-35,0)(60,60)
      \putnode{n1}{origin}{-10}{35}{\pscirclebox{1}}
      \putnode{n2}{n1}{-15}{-15}{\pscirclebox{4}}
      \putnode{n3}{n1}{0}{-30}{\pscirclebox{5}}
      \putnode{n4}{n2}{30}{0}{\pscirclebox{2}}
      \putnode{n5}{n3}{30}{0}{\pscirclebox{3}}

      \ncline[nodesep=-0.9]{->}{n1}{n2}
      \ncline[nodesep=-0.9]{->}{n1}{n4}
      \ncline[nodesep=-0.9]{->}{n2}{n3}
      \ncline[nodesep=-0.9]{->}{n4}{n3}
      \ncline[nodesep=-0.9]{->}{n4}{n5}

      \nccurve[nodesep=-0.3,angleA=-180,angleB=-180,ncurv=2]{->}{n3}{n1}
      \nccurve[nodesep=-0.7,angleA=45,angleB=45]{->}{n5}{n4}

      \putnode{lab}{n1}{30}{10}{%
        \begin{tabular}{l}
          Node 2 generates definition {\red d}. \\
          \onslide<2->{Other nodes ``empty'' w.r.t. {\red d}.} \\
          \onslide<3->{Does {\red d} reach node {\bf 4}?}
      \end{tabular}}
      \onslide<4->{\putnode{lo4}{n4}{0}{-5}{{\green d}}}
      \onslide<5->{\putnode{li5}{n5}{0}{ 5}{{\green d}}}
      \onslide<6->{\putnode{lo5}{n5}{0}{-5}{{\green d}}}
      \onslide<7->{\putnode{li3}{n3}{0}{ 5}{{\green d}}}
      \onslide<8->{\putnode{lo3}{n3}{0}{-5}{{\green d}}}
      \onslide<9->{\putnode{li1}{n1}{0}{ 5}{{\red d}}}
      \onslide<10->{\putnode{lo1}{n1}{0}{-5}{{\red d}}}
      \onslide<11->{\putnode{li4}{n4}{0}{ 5}{{\red d}}}
      \onslide<12->{\putnode{li2}{n2}{0}{ 5}{{\red d}}}
      \onslide<13->{\putnode{lo2}{n2}{0}{-5}{{\red d}}}
    \end{pspicture}
  \end{center}
}

\frame{
  \frametitle{Depth of a Flow Graph}
  \begin{itemize}[<+->]
  \item The {\bf depth} of a flow graph is the greatest number of retreating edges along any acyclic path.
  \item For RD, if we use DF order to visit nodes, we converge in depth+2 passes.
    \begin{itemize}
    \item Depth+1 passes to follow that number of increasing segments.
    \item 1 more pass to realize we converged.
    \end{itemize}
  \end{itemize}
}

\frame{
  \frametitle{Example: Depth = 2}
  \begin{center}
    \psset{unit=1mm}
    \begin{pspicture}(0,0)(80,30)
      \putnode{a0}{origin}{0}{10}{ }
      \putnode{b0}{a0}{20}{0}{ }
      \putnode{c0}{b0}{10}{0}{ }
      \putnode{d0}{c0}{20}{0}{ }
      \putnode{e0}{d0}{10}{0}{ }
      \putnode{f0}{e0}{20}{0}{ }
      \pause
      \ncline{->}{a0}{b0}
      \Bput{increasing}
      \pause
      \ncline[linestyle=dotted,dotsep=0.3]{->}{b0}{c0}
      \Aput{retreating}
      \pause
      \ncline{->}{c0}{d0}
      \Bput{increasing}
      \pause
      \ncline[linestyle=dotted,dotsep=0.3]{->}{d0}{e0}
      \Aput{retreating}
      \pause
      \ncline{->}{e0}{f0}
      \Bput{increasing}
    \end{pspicture}
  \end{center}
}

\frame{
  \frametitle{Similarly \ldots}
  \begin{itemize}[<+->]
  \item AE also works in depth+2 passes.
    \begin{itemize}
    \item Unavailability propagates along retreat-free node
      sequences in one pass.
    \end{itemize}
  \item So does LV if we use reverse of DF order.
    \begin{itemize}
    \item A use propagates backward along paths that do not use a
      retreating edge in one pass.
    \end{itemize}
  \end{itemize}
}

\frame{
  \frametitle{In General \ldots}
  \begin{itemize}
  \item The depth+2 bound works for any monotone bit-vector
    framework, as long as information only needs to propagate along
    acyclic paths.
    \begin{itemize}
    \item Example: if a definition reaches a point, it does so
      along an acyclic path.
    \end{itemize}
  \end{itemize}
}

\frame{
  \frametitle{Why Depth+2 is Good?}
  \begin{itemize}
  \item Normal control-flow constructs produce reducible flow
    graphs with the number of back edges at most the nesting
    depth of loops.
    \begin{itemize}
    \item Nesting depth tends to be small.
    \end{itemize}
  \end{itemize}
}

\frame{
  \frametitle{Example: Nested Loops}
  \begin{center}
    \psset{unit=1mm}
    \begin{pspicture}(0,0)(80,50)
      \putnode{a2}{origin}{10}{50}{%
        \pscirclebox[fillcolor=blue!20,fillstyle=solid]{\ \ \ }}
      \putnode{b2}{a2}{0}{-13}{%
        \pscirclebox[fillcolor=blue!20,fillstyle=solid]{\ \ \ }}
      \putnode{c2}{a2}{0}{-26}{%
        \pscirclebox[fillcolor=blue!20,fillstyle=solid]{\ \ \ }}
      \putnode{d2}{a2}{0}{-39}{%
        \pscirclebox[fillcolor=blue!20,fillstyle=solid]{\ \ \ }}
      \ncline[nodesep=-0.3]{->}{a2}{b2}
      \ncline[nodesep=-0.3]{->}{b2}{c2}
      \ncline[nodesep=-0.3]{->}{c2}{d2}
      \nccurve[nodesep=-0.1,angleA=-195,angleB=-165,
        linecolor=red,linewidth=.3]{->}{d2}{c2}
      \nccurve[nodesep=-0.1,angleA=-195,angleB=-165,
        linecolor=red,linewidth=.3]{->}{c2}{b2}
      \nccurve[nodesep=-0.1,angleA=-195,angleB=-165,
        linecolor=red,linewidth=.3]{->}{b2}{a2}
      \putnode{l2}{d2}{5}{-15}{%
        \begin{tabular}{p{45mm}}
          3 nested while loops.\\      \pause
          depth = 3.
      \end{tabular}}
      \pause
      \putnode{a1}{a2}{40}{0}{%
        \pscirclebox[fillcolor=blue!20,fillstyle=solid]{\ \ \ }}
      \putnode{b1}{a1}{0}{-13}{%
        \pscirclebox[fillcolor=blue!20,fillstyle=solid]{\ \ \ }}
      \putnode{c1}{a1}{0}{-26}{%
        \pscirclebox[fillcolor=blue!20,fillstyle=solid]{\ \ \ }}
      \putnode{d1}{a1}{0}{-39}{%
        \pscirclebox[fillcolor=blue!20,fillstyle=solid]{\ \ \ }}
      \ncline[nodesep=-0.3]{->}{a1}{b1}
      \ncline[nodesep=-0.3]{->}{b1}{c1}
      \ncline[nodesep=-0.3]{->}{c1}{d1}
      \nccurve[nodesep=-0.1,angleA=-195,angleB=-170,
        linecolor=red,linewidth=.3]{->}{d1}{a1}
      \nccurve[nodesep=-0.2,angleA=-195,angleB=-155,
        linecolor=red,linewidth=.3]{->}{c1}{a1}
      \nccurve[nodesep=-0.3,angleA=-195,angleB=-145,
        linecolor=red,linewidth=.3]{->}{b1}{a1}
      \putnode{l1}{d1}{20}{-15}{%
        \begin{tabular}{p{45mm}}
          3 nested do-while loops.\\ \pause
          depth = 1.
      \end{tabular}}
    \end{pspicture}
  \end{center}
}

\frame{
  \frametitle{Natural Loops}
  \begin{itemize}[<+->]
  \item The {\bf natural loop} of a back edge $a\rightarrow b$ is
    $\{b\}$ plus the set of nodes that can reach $a$ without
    going through $b$.
  \item {\bf Theorem:} two natural loops are either disjoint,
    identical, or nested.
  \item Proof: Discuss/Exercise
  \end{itemize}
}

\frame{
  \frametitle{Example: Natural Loops}
  \begin{center}
    \psset{unit=1mm}
    \begin{pspicture}(0,0)(60,30)
      %\psframe(-35,0)(60,60)
      \putnode{n1}{origin}{10}{35}{\pscirclebox{1}}
      \putnode{n2}{n1}{-15}{-15}{\pscirclebox{4}}
      \putnode{n3}{n1}{0}{-30}{\pscirclebox{5}}
      \putnode{n4}{n2}{30}{0}{\pscirclebox{2}}
      \putnode{n5}{n3}{30}{0}{\pscirclebox{3}}

      \onslide<3->{\putnode{lp1}{n2}{17}{0}{\psovalbox[framesep=25,fillcolor=red!10,fillstyle=solid]{\rule{25mm}{0pt}\rule{0pt}{5mm}}}}
      \onslide<2->{\putnode{lp1}{n4}{7}{-8}{\psovalbox[framesep=10,fillcolor=blue!10,fillstyle=solid]{\rule{5mm}{0pt}\rule{0pt}{8mm}}}}

      \putnode{n1}{origin}{10}{35}{\pscirclebox{1}}
      \putnode{n2}{n1}{-15}{-15}{\pscirclebox{4}}
      \putnode{n3}{n1}{0}{-30}{\pscirclebox{5}}
      \putnode{n4}{n2}{30}{0}{\pscirclebox{2}}
      \putnode{n5}{n3}{30}{0}{\pscirclebox{3}}


      \ncline[nodesep=-0.9]{->}{n1}{n2}
      \ncline[nodesep=-0.9]{->}{n1}{n4}
      \ncline[nodesep=-0.9]{->}{n2}{n3}
      \ncline[nodesep=-0.9]{->}{n4}{n3}
      \ncline[nodesep=-0.9]{->}{n4}{n5}
      \nccurve[nodesep=-0.3,angleA=-180,angleB=-180,ncurv=2]{->}{n3}{n1}
      \nccurve[nodesep=-0.7,angleA=45,angleB=45]{->}{n5}{n4}
      \onslide<2->{\putnode{lab5}{n5}{0}{-15}{Natural loop $3\rightarrow 2$}}
      \onslide<3->{\putnode{lab2}{n2}{0}{-35}{Natural loop $5\rightarrow 1$}}

    \end{pspicture}
  \end{center}
}

\frame{
  \frametitle{Reading Assignment}
  \begin{itemize}
  \item New Dragon Book (Aho, Lam, Sethi, Ullman)
    \begin{itemize}
    \item Chapter 9
    \end{itemize}
  \end{itemize}
}

\end{document}
