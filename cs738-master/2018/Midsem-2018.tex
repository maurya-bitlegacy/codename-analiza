\documentclass[12pt]{article}
\usepackage{graphicx}
\usepackage{graphics}
\usepackage{anysize}
\usepackage{fancyhdr}
\usepackage{pstricks}
\usepackage{pst-node}
\usepackage{pst-rel-points}
\usepackage{xspace}

\setlength{\headheight}{15pt}
\pagestyle{fancyplain}
\lhead{\fancyplain{}{{\em Page \#\/}\thepage}}
\chead{CS738}
\rhead{\fancyplain{}{\textit{Roll No:}\rule{30mm}{0pt}}}
\lfoot{}
\cfoot{}
\rfoot{}

\newcommand{\tbd}[1]{{\red #1}}
\newcommand{\answer}[1]{{{\blue #1}}}
%\renewcommand{\answer}[1]{\mbox{}}
\newcommand{\symbox}[2]{S#1\psframebox{#2}\phantom{S#1}}
\newcommand{\rsymbox}[2]{\phantom{S#1}\psframebox{#2} S#1}

\newcommand{\meet}{\ensuremath{\wedge}\xspace}

%% \textwidth 7in
%% \addtolength{\oddsidemargin}{-1in}
%% \textheight 9.5in
%% \addtolength{\topmargin}{-1in}
%% \setlength{\parindent}{0pt}
%% \setlength{\parskip}{0.5cm}
%% \topskip 0.0in
\thispagestyle{empty}

\begin{document}

{\Large\bf NAME: \hspace*{3in} ROLL NO:}

\begin{center}
CS738: Advanced Compiler Optimizations \\
Mid Semester Examination, 2018-19 I\\
Max Time: 2 Hours  \hfill Max Marks: {95} \\

{\bf NOTE: }
\begin{itemize}
\item There are total {\bf 3} questions on {\bf 3 pages}.
\item Write your name and roll number on the question paper and the
  answer book.
\item Presenting your answers properly is your
  responsibility. You lose credit if you can not present your
  ideas clearly, and in proper form. Please DO NOT come back
  for re-evaluation saying, ``What I actually meant was
  \ldots''.
\item Be precise and write clearly. Remember that somebody
  has to read it to evaluate!
\end{itemize}
\hrule
\end{center}


\newcommand{\pt}{\ensuremath{\pi}}
\newcommand{\fgentry}{{\sf Entry}}
\newcommand{\fgexit}{{\sf Exit}}
\begin{center}
  \Large\bf Notations
\end{center}
\newcommand{\indef}[2]{\ensuremath{{\sf BeforeIN}(#1, #2)}}
\newcommand{\outdef}[2]{\ensuremath{{\sf BeforeOut}(#1, #2)}}
\newcommand{\inuse}[2]{\ensuremath{{\sf AfterIN}(#1, #2)}}
\newcommand{\outuse}[2]{\ensuremath{{\sf AfterOut}(#1, #2)}}
\newcommand{\inacc}[2]{\ensuremath{{\sf AccIN}(#1, #2)}}
\newcommand{\outacc}[2]{\ensuremath{{\sf AccOUT}(#1, #2)}}
\newcommand{\inaccset}[2]{\ensuremath{{\sf AccSetIN}(#1, #2)}}
\newcommand{\outaccset}[2]{\ensuremath{{\sf AccSetOUT}(#1, #2)}}
\newcommand{\IN}[1]{\ensuremath{{\sf IN}(#1)}}
\newcommand{\OUT}[1]{\ensuremath{{\sf OUT}(#1)}}
\newcommand{\Pred}[1]{\ensuremath{{\sf PRED}(#1)}}
\newcommand{\Succ}[1]{\ensuremath{{\sf SUCC}(#1)}}
\newcommand{\cpath}{\ensuremath{\stackrel{+}{\longrightarrow}}}
\newcommand{\idfp}{\ensuremath{\mbox{DF}^+}}

\begin{itemize}
\item CFG stands for control flow graph.
\item \IN{S} denotes the program point before the statement
  $S$. \OUT{S} denotes the program point after
  the statement of $S$.
\item \Pred{S} denotes the set of predecessors, and \Succ{S}
  denotes the set of successors of $S$. %in the CFG.
\item In a CFG, $x \cpath y$ denotes a path from
  node $x$ to node $y$, having one or more edges. Both $x$ and
  $y$ are considered to be a part of the path.
\item \idfp($\varphi$) denotes the Iterated Dominance Frontier of
  the set of CFG nodes $\varphi$.
\end{itemize}
\hrule
\vfill
\begin{enumerate}
%%%%%%%%%%%%%%%%%%%%%%%%%%%%%%%%%%%%%%%%%%%%%%%%%%%%%%%%%%%%%%%%%%%%%
\item Prove the following statement: \hfill[15\answer{[5+10]}]

  For any non-null path $p\!\!\!:\!\! X\!\!\cpath Z$ in a CFG, there exists a node \linebreak
  $X' \in \{X\} \cup \idfp(\{X\})$ on $p$ that dominates
  $Z$. Moreover, unless $X$ dominates every node on p, the node
  $X'$ can be chosen in $\idfp(\{X\})$.

\answer{
  \begin{enumerate}
  \item {\bf $X$ dominates every node in $p$}. Clearly $X$
    dominates $Z$.
  \item {\bf $X$ does not dominate every node in $p$}. Suppose the sequence of nodes in the path $p$ is $n_0 (=X), n_1, n_2, \ldots, n_k (=Z)$. Since $X$ does not dominate all nodes in $p$, some of the nodes in $p$ will be in $\idfp(\{X\})$ {\bf (WHY?)}. Let $n_j$ be the node in $\idfp(\{X\})$ such that it has the highest value of $j$. We claim that $X' = n_j$, i.e., $n_j$ dominates $Z$.
  Suppose $n_j$ does not dominate $Z = (n_k)$. Then, $\exists i, j < i \leq k$ such that $n_j$ does not dominate $n_i$. Choose smallest such $i$. We have, parent of $n_i$ dominated by $n_j$, but $n_i$ is not (strictly) dominated by $n_j$. This gives us:
  \begin{eqnarray*}
    &&n_i \in \mbox{DF}(\{n_j\})\\
       &\Rightarrow& n_i \in \idfp(\{n_j\})\\
       &\Rightarrow& n_i \in \idfp(\{X\})
  \end{eqnarray*}
But this contradicts the fact that $j$ is the largest index such that $n_j \in \idfp(\{X\}).$
  \end{enumerate}
}

\clearpage
%%%%%%%%%%%%%%%%%%%%%%%%%%%%%%%%%%%%%%%%%%%%%%%%%%%%%%%%%%%%%%%%%%%%%
\item\label{ques1} {\bf Shortest Use Distance of A Definition.}  \hfill[35\answer{[25+8+2]}]

  Let a definition $d$ define a variable $x$ at a program point $\pt_d$.  Let program point $\pt_u$ contain a use of $x$ on some path from $\pt_d$ to $\fgexit$, such that $x$ is not redefined between  $\pt_d$ and  $\pt_u$. The number of instructions between $\pt_d$ and $\pt_u$ is a {\bf use distance} of $d$. If there is no use of $x$ corresponding to $d$ on some path from $\pt_d$ to $\fgexit$, then the use distance on that path is $\infty$.

  The {\bf shortest use distance (SUD)} of $d$ is defined as the minimum over all use distances of $d$. 
  
  Figure~\ref{fig:accessrange} shows an example program CFG, having variables A, B and C. SUDs for various definitions for this example are:
  \begin{center}
    \begin{tabular}{|c|c|c|}
      \hline
      Stmt& Var& SUD \\ \hline\hline
      S1 & A & $\infty$ \\ \hline
      S2 & A & $2$ \\ \hline
      S3 & B & $1$ \\ \hline
      S6 & C & $1$ \\ \hline
    \end{tabular}
  \end{center}
\begin{figure}[h!]
\centering
%\includegraphics*[scale=0.59]{AccessRange.eps}
\begin{center}
\psset{xunit=1mm,yunit=.6mm}
\begin{pspicture}(0,10)(150,-130)
  %\psframe(0,10)(150,-160)
  \putnode{b0}{origin}{75}{5}{\symbox{0}{\fgentry}}
  \putnode{b1}{b0}{0}{-20}{\symbox{1}{$A = \ldots$}}
  \putnode{b2}{b1}{0}{-20}{\rsymbox{2}{$A = \ldots$}}
  \putnode{b3}{b2}{0}{-20}{\rsymbox{3}{$B = \ldots$}}
  \putnode{b4}{b3}{-20}{-20}{\rsymbox{4}{$\ldots = B$}}
  \putnode{b5}{b3}{20}{-20}{\symbox{5}{$\ldots = A$ }}
  \putnode{b6}{b5}{0}{-20}{\symbox{6}{$C = \ldots$}}
  \putnode{b7}{b6}{0}{-20}{\symbox{7}{$\ldots = C$}}
  \putnode{b8}{b7}{0}{-20}{\symbox{8}{\fgexit}}

  \ncline{->}{b0}{b1}
  \ncline{->}{b1}{b2}
  \ncline{->}{b2}{b3}
  \ncline{->}{b3}{b4}
  \ncline{->}{b3}{b5}
  \ncline{->}{b5}{b6}
  \ncline{->}{b6}{b7}
  \ncline{->}{b7}{b8}
  \nccurve[angleA=215,angleB=155,nodesep=0,ncurv=1.2]{->}{b4}{b2}
\end{pspicture}
\end{center}

\caption{An Example Program\label{fig:accessrange}}
\end{figure}

\begin{enumerate}
\item Design a data-flow analysis to compute SUDs for definitions
  present in a program. Recall that you have to talk about the 4
  components $\langle D, S, \meet, F\rangle$. Describe the lattice
  $\langle S, \meet\rangle$ in details, with the help of a lattice
  diagram. You also need to describe flow functions for various
  statements of interest.
\item Is your analysis guaranteed to terminate? Justify.
\item Give one application of this analysis.
\end{enumerate}
%%%%%%%%%%%%%%%%%%%%%%%%%%%%%%%%%%%%%%%%%%%%%%%%%%%%
\item Consider the following flow graph: \hfill[45\answer{[5+15+25]}]
\begin{center}
\scalebox{1.25}{    
\psset{xunit=1.3mm, yunit=1.3mm}
\begin{pspicture}(-9,0)(95,-96)
%  \psframe(-9,0)(95,-107)
%  \putnode{c0}{origin}{0}{-20}{}
  \putnode{b1}{origin}{38}{5}{1\psframebox[framesep=2pt]{
      \begin{tabular}{@{}l@{}}
        $a = \ldots$\\
        $b = \ldots$
      \end{tabular}
  }}
  \putnode{b2}{b1}{-22}{-12}{2\psframebox[framesep=2pt]{
      \begin{tabular}{@{}l@{}}
        \rule{10mm}{0mm}\rule{0mm}{5mm}
      \end{tabular}
  }}
  \putnode{b3}{b2}{22}{0}{3\psframebox[framesep=2pt]{
      \begin{tabular}{@{}l@{}}
        \rule{10mm}{0mm}\rule{0mm}{5mm}        
      \end{tabular}
  }}
  \putnode{b4}{b3}{22}{0}{4\psframebox[framesep=2pt]{
      \begin{tabular}{@{}l@{}}
        \rule{10mm}{0mm}\rule{0mm}{5mm}
      \end{tabular}
  }}
  \putnode{b5}{b2}{-26}{-12}{5\psframebox[framesep=2pt]{
      \begin{tabular}{@{}l@{}}
        \rule{10mm}{0mm}\rule{0mm}{5mm}
      \end{tabular}
  }}
  \putnode{b6}{b5}{32}{0}{6\psframebox[framesep=2pt]{
      \begin{tabular}{@{}l@{}}
        \rule{10mm}{0mm}\rule{0mm}{5mm}
      \end{tabular}
  }}
  \putnode{b7}{b6}{22}{0}{7\psframebox[framesep=2pt]{
      \begin{tabular}{@{}l@{}}
        \rule{10mm}{0mm}\rule{0mm}{5mm}
      \end{tabular}
  }}
  \putnode{b8}{b7}{11}{-12}{8\psframebox[framesep=2pt]{
      \begin{tabular}{@{}l@{}}
        $a = \ldots$
      \end{tabular}
  }}
  \putnode{b9}{b8}{22}{0}{9\psframebox[framesep=2pt]{
      \begin{tabular}{@{}l@{}}
        $a = \ldots$
      \end{tabular}
  }}
  \putnode{b11}{b8}{-22}{0}{11\psframebox[framesep=2pt]{
      \begin{tabular}{@{}l@{}}
        \rule{10mm}{0mm}\rule{0mm}{5mm}
      \end{tabular}
  }}
  \putnode{b10}{b11}{-32}{0}{10\psframebox[framesep=2pt]{
      \begin{tabular}{@{}l@{}}
        \rule{10mm}{0mm}\rule{0mm}{5mm}
      \end{tabular}
  }}
  \putnode{b12}{b8}{11}{-12}{12\psframebox[framesep=2pt]{
      \begin{tabular}{@{}l@{}}
        \rule{10mm}{0mm}\rule{0mm}{5mm}
      \end{tabular}
  }}
  \putnode{b17}{b12}{-11}{-12}{17\psframebox[framesep=2pt]{
      \begin{tabular}{@{}l@{}}
        \rule{10mm}{0mm}\rule{0mm}{5mm}
      \end{tabular}
  }}
  \putnode{b16}{b17}{-33}{0}{16\psframebox[framesep=2pt]{
      \begin{tabular}{@{}l@{}}
        \rule{10mm}{0mm}\rule{0mm}{5mm}
      \end{tabular}
  }}
  \putnode{b13}{b17}{-27}{-12}{13\psframebox[framesep=2pt]{
      \begin{tabular}{@{}l@{}}
        \rule{10mm}{0mm}\rule{0mm}{5mm}
      \end{tabular}
  }}
  \putnode{b14}{b13}{0}{-13}{14\psframebox[framesep=2pt]{
      \begin{tabular}{@{}l@{}}
        \rule{10mm}{0mm}\rule{0mm}{5mm}
      \end{tabular}
  }}
  \putnode{b15}{b14}{-20}{-14}{15\psframebox[framesep=2pt]{
      \begin{tabular}{@{}l@{}}
        \rule{10mm}{0mm}\rule{0mm}{5mm}
      \end{tabular}
  }}
  
  \ncline[nodesepA=-1.5mm]{->}{b1}{b2}
  \ncline{->}{b1}{b3}
  \ncline[nodesepB=-1.1mm]{->}{b1}{b4}
  \ncline[nodesepA=-1.5mm]{->}{b2}{b5}
  \ncline{->}{b2}{b10}
  \ncline{->}{b3}{b6}
  \ncline{->}{b3}{b7}
  \ncline{->}{b4}{b8}
  \ncline{->}{b4}{b9}
  \ncline{->}{b5}{b10}
  \ncline{->}{b6}{b11}
  \ncline{->}{b7}{b11}
  \ncline{->}{b8}{b12}
  \ncline{->}{b9}{b12}
  \ncline{->}{b10}{b16}
  \ncline{->}{b10}{b15}
  \ncline[offsetB=.5]{->}{b11}{b13}
  \ncline{->}{b12}{b17}
  \ncline{->}{b13}{b14}
  \nccurve[angleA=-135,angleB=135,ncurv=2]{->}{b14}{b13}
  \nccurve[angleA=-90]{->}{b14}{b15}
  \ncline{->}{b16}{b13}
  \nccurve[angleA=-90,angleB=45,ncurv=1]{->}{b17}{b14}
\end{pspicture}
}
\end{center}
Block \#1 is the $\fgentry$ block.
\begin{enumerate}
\item Draw the dominator tree for the graph.
\item Calculate the dominance frontier for each
  block.
\item Convert the flow graph to minimal SSA form. Show the
  important steps in conversion.
\end{enumerate}

\end{enumerate}
\end{document}

