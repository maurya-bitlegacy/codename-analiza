\documentclass[12pt]{article}
\usepackage{graphicx}
\usepackage{graphics}
\usepackage{anysize}
\usepackage{fancyhdr}
\usepackage{pstricks}
\usepackage{pst-node}
\usepackage{pst-rel-points}

\setlength{\headheight}{15pt}
\pagestyle{fancyplain}
\lhead{\fancyplain{}{{\em Page \#\/}\thepage}}
\chead{CS618}
\rhead{\fancyplain{}{\textit{Roll No:}\rule{30mm}{0pt}}}
\lfoot{}
\cfoot{}
\rfoot{}

\newcommand{\tbd}[1]{{\red #1}}
\newcommand{\answer}[1]{{{\blue #1}}}
%\renewcommand{\answer}[1]{\mbox{}}
\newcommand{\symbox}[2]{B#1\psframebox{#2}\phantom{B#1}}
\newcommand{\rsymbox}[2]{\phantom{B#1}\psframebox{#2} B#1}


%% \textwidth 7in
%% \addtolength{\oddsidemargin}{-1in}
%% \textheight 9.5in
%% \addtolength{\topmargin}{-1in}
%% \setlength{\parindent}{0pt}
%% \setlength{\parskip}{0.5cm}
%% \topskip 0.0in
\thispagestyle{empty}

\begin{document}

{\Large\bf NAME: \hspace*{3in} ROLL NO:}

\begin{center}
CS618: Program Analysis \\
Mid Semester Examination, 2016-17 I\\
Max Time: 2 Hours  \hfill Max Marks: 125 \\

{\bf NOTE: }
\begin{itemize}
\item There are total {\bf 5} questions on {\bf 4 pages}
\item Write your name and roll number on the question paper and the
  answer book.
\item No explanations will be provided. In case of a doubt, make
  suitable assumptions and justify.

\item Presenting your answers properly is your
  responsibility. You lose credit if you can not present your
  ideas clearly, and in proper form. Please DO NOT come back
  for re-evaluation saying, ``What I actually meant was
  \ldots''.

\item Be precise and write clearly. Remember that somebody
  has to read it to evaluate!

\end{itemize}
\hrule
\end{center}
%\clearpage
%%%%%%%%%%%%%%%%%%%%%%%%%%%%%%%%%%%%%%%%%%%%%%%%%%%%
%\thispagestyle{empty} \vspace{-5mm}
%\includegraphics[angle=90,scale=.99]{Endsem-2011.Q2.eps}
%\clearpage
\begin{enumerate}
%%%%%%%%%%%%%%%%%%%%%%%%%%%%%%%%%%%%%%%%%%%%%%%%%%%%
\item Consider a flow graph G with a unique entry node ENTRY that
  dominates all nodes of G. Prove that every node in G except
  ENTRY has a unique {\em immediate dominator}.  By definition,
  immediate dominator of a node $n$ is the {\em closest strict
    dominator} of $n$.  \hfill[10] \answer{
  \begin{itemize}
  \item Every node except ENTRY has at least one strict
    dominator. Consider a node Z that has more than one strict
    dominators. Consider two such dominators X and Y. Then, it
    can be proved that either X dominates Y or Y dominates X
    (*proof below). Thus, it possible to find {\em least} element
    among all strict dominators of Z. This element is the desired
    immediate dominator.
  \item * Consider a cycle free path from
    ENTRY$\rightarrow$Z. Because both X and Y strictly
    dominate Z, they must occur on this path. WLOG, assume
    that X occurs before Y in this path.  Thus the path is:
    Entry $\rightarrow$ X $\rightarrow$ Y $\rightarrow$ Z. We
    will prove that X must dominate Y.

    Assume the contrary (X does not dominate Y). Then, we
    have a path Entry $\Rightarrow Y$ free of X. But then,
    Entry $\Rightarrow$ Y $\rightarrow$ Z is a path to Z free
    of X. But that also contradicts the fact that X dominates
    Z.
  \end{itemize}
}

\newcommand{\df}{\mbox{\rm df}}
\item The original definition of {\em Dominance Frontier} (\df) is:
A node $m$ is in $\df(n)$ if
\begin{enumerate}
\item $n$ dominates a predecessor of $m$ in the flow graph, and
\item $n$ does not strictly dominate $m$
\end{enumerate}
Dr Dominoz thinks the following modified definition of $\df(n)$ is equivalent as far as computation of SSA form is concerned: A node $m$ is in $\df(n)$ if
\begin{enumerate}
\item $n$ dominates a predecessor of $m$ in the flow graph, and
\item $n$ does not dominate $m$
\end{enumerate}
i.e., Dr Dominoz has dropped the term ``strictly'' from the
definition.

You jobs  is to either {\bf  prove that  Dr Dominoz is  right} or
show that  he is {\bf  wrong, by  giving a counter  example}. The
proof must work for any  arbitrary CFG, while the counter example
must  show   an  incorrect  SSA   form  being  generated   for  a
CFG.\hfill[10]

%%%%%%%%%%%%%%%%%%%%%%%%%%%%%%%%%%%%%%%%%%%%%%%%%%%%
\clearpage
\newcommand{\px}{\mbox{\bf x}}
\item Consider the following extensions to our 3-address code
  language:
  \begin{itemize}
  \item {\tt \px.lock}: ``locks'' a variable \px
  \item {\tt \px.unlock}: ``unlocks'' a variable \px
  \item {\tt \px.secureComp} does some ``secure computation'' on \px
  \end{itemize}

  In this language, a ``secure computation'' on a variable \px\
  is allowed only when it is locked (\px.lock executed before
  \px.secureComp, without an intervening \px.unlock). We call
  such secureComp {\bf safe}, otherwise it is {\bf unsafe}.

  The language obviously contains basic constructs like {\em
    assignment} statements, {\em goto} statements, and
  conditionals ({\em if-goto}). Following semantic properties
  hold:
 \begin{itemize}
 \item All variables are {\bf unlocked} at the {\em entry}
 \item  {\bf lock}  and  {\bf unlock}  operations are  idempotent
   (Locking a locked variable is allowed, but it has no effect on
   the  lock-status  of  the  variable.  Similarly,  unlocking  an
   unlocked variable is also allowed)
 \end{itemize}
 Here are couple of sample programs, {\tt PROGRAM-1} is valid and {\tt PROGRAM-2} is invalid.
\begin{verbatim}
   // PROGRAM-1                  // PROGRAM-2
     c = 5                         c = 5 
     n = 0                         n = 0
     t = c > 0                     t = c > 0
     if t goto L1                  if t goto L1
     n.lock                          n.lock   
     n.secureComp                    c.secureComp // BAD, c not locked
     goto L2                         c.unlock  
                                     goto L2
   L1:                             L1:
     c.lock                          n.lock
     c.secureComp                    n.secureComp
     d = e - 5;                      d  = e - 5;
     c.unlock                        c.lock
   L2:                             L2:
     n.lock                          n.secureComp // OK, n locked on
                                                  // all paths
     n.secureComp                    c.secureComp // BAD,  c may not
                                                  // be locked 
     n.unlock                        n.unlock
     c.unlock                        c.unlock
\end{verbatim}

 Design an {\bf intraprocedural} data flow
 analysis framework to mark unsafe secure computations
 (``secureComp''). In particular,
 \begin{enumerate}
 \item Draw the lattice for the
   framework, and describe it briefly.\hfill[5]
 \item Describe the meet operator ($\wedge$).\hfill[5]
 \item Describe intuitively the meaning of the top and the
   bottom elements. \hfill[5]
 \item Define the flow functions  for statements. You do not need
   to   list  all   types   of  statements,   but  use   suitable
   representatives (for  e.g. {\tt  x op  y} to  represent binary
   operators).\hfill[10]
 \item Is your framework {\em Forward} or {\em Backward}?
   Justify your answer. Also describe the {\em BoundaryInfo}
   (initialization information at the boundary of the flow
   graph). \hfill[2+3]
 \end{enumerate}
 \answer{
   \begin{enumerate}
\item Lattice:
  \begin{center}
    \psset{unit=1mm}
    \begin{pspicture}(0,0)(50,50)
      %\psframe(0,0)(50,50)
      \putnode{b0}{origin}{25}{45}{$\top$}
      \putnode{b1}{b0}{-20}{-20}{int}
      \putnode{b2}{b0}{20}{-20}{bool}
      \putnode{b3}{b0}{0}{-40}{$\bot$}        
      \ncline{b0}{b1}
      \ncline{b0}{b2}
      \ncline{b2}{b3}
      \ncline{b1}{b3}      
    \end{pspicture}
  \end{center}
\item meet: 
  \begin{center}
    \begin{tabular}{c||c|c|c|c}
      $\wedge$ & $\top$ & int    & bool   & $\bot$ \\ \hline \hline
      $\top$   & $\top$ & int    & bool   & $\bot$ \\ \hline
      int      & int    & int    & $\bot$ & $\bot$ \\ \hline
      bool     & bool   & $\bot$ & bool   & $\bot$ \\ \hline
      $\bot$   & $\bot$ & $\bot$ & $\bot$ & $\bot$ \\ \hline
    \end{tabular}
  \end{center}
\item $\top$ represents no type inferred yet. $\bot$
  represents conflicting types inferred (more than one type
  for a variable).
\item Flow functions. (This is one of the many possible solutions)
  \begin{enumerate}
  \item z = x == y:
    \begin{eqnarray*}
      Out(z) &=& In(z) \wedge bool \\
      Out(x) &=& In(x) \wedge In(y) \\
      Out(y) &=& In(x) \wedge In(y) 
    \end{eqnarray*}

  \item z = x + y:
    \begin{eqnarray*}
      Out(z) &=& In(z) \wedge int \\
      Out(x) &=& In(x) \wedge int \\
      Out(y) &=& In(y) \wedge int 
    \end{eqnarray*}

  \item z = x \&\& y:
    \begin{eqnarray*}
      Out(z) &=& In(z) \wedge bool \\
      Out(x) &=& In(x) \wedge bool \\
      Out(y) &=& In(y) \wedge bool
    \end{eqnarray*}

  \item z = int\_constant:
    \begin{eqnarray*}
      Out(z) &=& In(z) \wedge int \\
    \end{eqnarray*}

  \item z = bool\_constant:
    \begin{eqnarray*}
      Out(z) &=& In(z) \wedge bool \\
    \end{eqnarray*}

  \item z = x:
    \begin{eqnarray*}
      Out(z) &=& In(z) \wedge In(x)  \\
      Out(x) &=& In(z) \wedge In(x)  \\
    \end{eqnarray*}
  \end{enumerate}
\item My framework is forward (but yours could be backward,
  depends on the flow functions!). Out is computed in terms
  of In. 
    \begin{eqnarray*}
      \mbox{\em BoundaryInfo} &=& BI, \mbox{ s.t. } BI(x) =
      \top\       \forall x \\
    \end{eqnarray*} 
\end{enumerate}
}


%%%%%%%%%%%%%%%%%%%%%%%%%%%%%%%%%%%%%%%%%%%%%%%%%%%%
%%\clearpage
\item Consider the following flow graph:

\begin{center}
\psset{unit=1mm}
\begin{pspicture}(0,0)(150,-120)
  \psframe(0,10)(150,-120)
  \putnode{b1}{origin}{75}{5}{\symbox{1}{ENTRY}}
  \putnode{b2}{b1}{0}{-20}{\symbox{2}{x = 2}}
  \putnode{b3}{b2}{-30}{-20}{\symbox{3}{y = 3}}
  \putnode{b4}{b2}{30}{-20}{\symbox{4}{x = x + 1}}
  \putnode{b5}{b3}{-15}{-20}{\symbox{5}{z = x + y}}
  \putnode{b6}{b3}{15}{-20}{\symbox{6}{\phantom{w = y + 3}}}
  \putnode{b7}{b4}{-15}{-20}{\symbox{7}{y = y + 1}}
  \putnode{b8}{b4}{15}{-20}{\symbox{8}{\phantom{y = y + 1}}}
  \putnode{b9}{b5}{15}{-20}{\symbox{9}{\phantom{w = 2 * x}}}
  \putnode{b10}{b6}{15}{-20}{\symbox{10}{ x = z - 1}}
  \putnode{b11}{b7}{15}{-20}{\symbox{11}{y = y + x}}
  \putnode{b12}{b10}{0}{-20}{\symbox{12}{\phantom{y = y + 1}}}
  \putnode{b15}{b12}{0}{-20}{\symbox{13}{EXIT}}

  \ncline{->}{b1}{b2}
  \ncline{->}{b2}{b3}
  \ncline{->}{b2}{b4}
  \ncline{->}{b3}{b5}
  \ncline{->}{b3}{b6}
  \ncline{->}{b4}{b7}
  \ncline{->}{b4}{b8}
  \ncline{->}{b5}{b9}
  \ncline{->}{b6}{b9}
  \ncline{->}{b6}{b10}
  \ncline{->}{b7}{b10}
  \ncline{->}{b8}{b11}
  \ncline{->}{b10}{b12}
  \ncline{->}{b11}{b12}
  \ncline{->}{b13}{b15}
  \ncline{->}{b12}{b15}

  \nccurve[angleA=-155,angleB=135,nodesep=-.1,ncurv=2]{->}{b9}{b3}
  \nccurve[angleA=-35,angleB=25,nodesep=-.1,ncurv=1.2]{->}{b12}{b8}
\end{pspicture}
\end{center}

\begin{enumerate}
\item Draw the dominator tree for the graph. \hfill[10]
\item Calculate the dominance frontier for each
  block. \hfill[15] 
\item Calculate the iterated dominance frontiers for the nodes containing the definitions of x, y and z. Assume that
  ENTRY node (B1) contains implicit definitions of x, y, z as
  {\tt undef}.
\hfill[3*5 = 15] 
\item Convert the flow graph to minimal SSA form.  
\hfill[10]
\end{enumerate}

\answer{
\clearpage
  Dominator Tree:
\begin{center}
\psset{unit=1mm}
\begin{pspicture}(0,0)(150,-90)
  %\psframe(0,0)(150,-90)
  \putnode{b1}{origin}{75}{-5}{\symbox{1}{ENTRY}}
  \putnode{b2}{b1}{0}{-20}{\symbox{2}{x = 2}}
  \ncline{->}{b1}{b2}

  \putnode{b3}{b2}{-50}{-20}{\symbox{3}{y = 3}}
  \putnode{b4}{b3}{20}{0}{\symbox{4}{x = x + 1}}
  \putnode{b10}{b4}{20}{0}{\symbox{10}{ x = z - 1}}
  \putnode{b11}{b10}{20}{0}{\symbox{11}{y = y + x}}
  \putnode{b12}{b11}{20}{0}{\symbox{12}{\phantom{y = y + 1}}}
  \putnode{b15}{b12}{0}{-20}{\symbox{13}{EXIT}}
  \ncline{->}{b2}{b3}
  \ncline{->}{b2}{b4}
  \ncline{->}{b2}{b10}
  \ncline{->}{b2}{b11}
  \ncline{->}{b2}{b12}
  \ncline{->}{b2}{b15}

  \putnode{b5}{b3}{-35}{-20}{\symbox{5}{z = x + y}}
  \putnode{b6}{b5}{15}{0}{\symbox{6}{\phantom{w = y + 3}}}
  \putnode{b9}{b6}{15}{0}{\symbox{9}{\phantom{w = 2 * x}}}
  \ncline{->}{b3}{b5}
  \ncline{->}{b3}{b6}
  \ncline{->}{b3}{b9}

  \putnode{b7}{b4}{-10}{-20}{\symbox{7}{y = y + 1}}
  \putnode{b8}{b7}{20}{0}{\symbox{8}{\phantom{y = y + 1}}}
  \ncline{->}{b4}{b7}
  \ncline{->}{b4}{b8}

\end{pspicture}
\end{center}

Dominance Frontier:
%\begin{center}
  \begin{tabular}{r|l}
    Node & DF \\ \hline \hline
     1 & \{\} \\ \hline
     2 & \{\} \\ \hline
     3 & \{3,10,13\} \\ \hline
     4 & \{10,11\} \\ \hline
     5 & \{3,9\} \\ \hline
     6 & \{9,10\} \\ \hline
     7 & \{10\} \\ \hline
     8 & \{11\} \\ \hline
     9 & \{13\} \\ \hline
    10 & \{12\} \\ \hline
    11 & \{12\} \\ \hline
    12 & \{11,13,15\} \\ \hline
    13 & \{15\} \\ \hline
    14 & \{11,15\} \\ \hline
    15 & \{\} \\ \hline
  \end{tabular}
%\end{center}

\begin{center}
  ENTRY node (1) contains implicit definitions of x, y, z.

  \begin{tabular}{c|c|l} \hline
  Var & Defs & Iterated Dom Frontier of Defs \\ \hline \hline
  x & 1, 2, 4, 10 & \{ 10, 11, 12, 13, 15 \} \\ \hline
  y & 1, 3, 7, 11 & \{  3, 10, 11, 12, 13, 15 \} \\ \hline
  z & 1, 5 &        \{  3,  9, 10, 11, 12, 13, 15 \} \\ \hline
  \end{tabular}
\end{center}
\begin{center}
\psset{unit=1mm}
\begin{pspicture}(0,0)(150,-160)
  \psframe(0,0)(150,-160)
  \putnode{b1}{origin}{75}{-10}{\symbox{1}{ENTRY [x, y, z]}}
  \putnode{b2}{b1}{0}{-20}{\symbox{2}{x = 2}}
  \putnode{b3}{b2}{-30}{-20}{\symbox{3}{
      \renewcommand{\arraystretch}{.1}
      \begin{tabular}{@{}l@{}}
        z = $\phi$(z, z) \\
        y = $\phi$(y, y) \\
        y = 3 \\
      \end{tabular}
  }}
  \putnode{b4}{b2}{30}{-20}{\symbox{4}{x = x + 1}}
  \putnode{b5}{b3}{-15}{-20}{\symbox{5}{z = x + y}}
  \putnode{b6}{b3}{15}{-20}{\symbox{6}{\phantom{w = y + 3}}}
  \putnode{b7}{b4}{-15}{-20}{\symbox{7}{y = y + 1}}
  \putnode{b8}{b4}{15}{-20}{\symbox{8}{\phantom{y = y + 1}}}
  \putnode{b9}{b5}{15}{-20}{\symbox{9}{
      \renewcommand{\arraystretch}{.1}
      \begin{tabular}{@{}l@{}}
        z = $\phi$(z, z)
      \end{tabular}
  }}
  \putnode{b10}{b6}{15}{-20}{\symbox{10}{ 
      \renewcommand{\arraystretch}{.1}
      \begin{tabular}{@{}l@{}}
        z = $\phi$(z, z) \\
        y = $\phi$(y, y) \\
        x = $\phi$(x, x) \\
        x = z - 1 \\
      \end{tabular}
  }}
  \putnode{b11}{b7}{15}{-20}{\symbox{11}{
      \renewcommand{\arraystretch}{.1}
      \begin{tabular}{@{}l@{}}
        z = $\phi$(z, z) \\
        y = $\phi$(y, y) \\
        x = $\phi$(x, x) \\
        y = y + x \\
      \end{tabular}
  }}
  \putnode{b12}{b10}{0}{-20}{\symbox{12}{
      \renewcommand{\arraystretch}{.1}
      \begin{tabular}{@{}l@{}}
        z = $\phi$(z, z) \\
        y = $\phi$(y, y) \\
        x = $\phi$(x, x) \\
      \end{tabular}
  }}
  \putnode{b15}{b12}{0}{-20}{\symbox{13}{
      \renewcommand{\arraystretch}{.1}
      \begin{tabular}{@{}l@{}}
        z = $\phi$(z, z) \\
        y = $\phi$(y, y) \\
        x = $\phi$(x, x) \\
        EXIT \\
      \end{tabular}
  }}

  \ncline{->}{b1}{b2}
  \ncline{->}{b2}{b3}
  \ncline{->}{b2}{b4}
  \ncline{->}{b3}{b5}
  \ncline{->}{b3}{b6}
  \ncline{->}{b4}{b7}
  \ncline{->}{b4}{b8}
  \ncline{->}{b5}{b9}
  \ncline{->}{b6}{b9}
  \ncline{->}{b6}{b10}
  \ncline{->}{b7}{b10}
  \ncline{->}{b8}{b11}
  \ncline{->}{b10}{b12}
  \ncline{->}{b11}{b12}
  \ncline{->}{b12}{b15}

  \nccurve[angleA=-155,angleB=135,nodesep=-.1,ncurv=2]{->}{b9}{b3}
  \nccurve[angleA=-35,angleB=25,nodesepB=-3,ncurv=1.2]{->}{b12}{b8}
\end{pspicture}
\end{center}

\begin{center}
\psset{unit=1mm}
\begin{pspicture}(0,0)(150,-160)
  \psframe(0,0)(150,-160)
  \putnode{b1}{origin}{75}{-10}{\symbox{1}{ENTRY [x0, y0, z0]}}
  \putnode{b2}{b1}{0}{-20}{\symbox{2}{x1 = 2}}
  \putnode{b3}{b2}{-30}{-20}{\symbox{3}{
      \renewcommand{\arraystretch}{.1}
      \begin{tabular}{@{}l@{}}
        z1 = $\phi$(z2, z0) \\
        y1 = $\phi$(y2, y0) \\
        y2 = 3 \\
      \end{tabular}
  }}
  \putnode{b4}{b2}{30}{-20}{\symbox{4}{x2 = x1 + 1}}
  \putnode{b5}{b3}{-15}{-20}{\symbox{5}{z2 = x1 + y2}}
  \putnode{b6}{b3}{15}{-20}{\symbox{6}{\phantom{w = y + 3}}}
  \putnode{b7}{b4}{-15}{-20}{\symbox{7}{y3 = y0 + 1}}
  \putnode{b8}{b4}{15}{-20}{\symbox{8}{\phantom{y = y + 1}}}
  \putnode{b9}{b5}{10}{-20}{\symbox{9}{
      \renewcommand{\arraystretch}{.1}
      \begin{tabular}{@{}l@{}}
        z3 = $\phi$(z2, z1)
      \end{tabular}
  }}
  \putnode{b10}{b6}{15}{-20}{\symbox{10}{ 
      \renewcommand{\arraystretch}{.1}
      \begin{tabular}{@{}l@{}}
        z4 = $\phi$(z1, z0) \\
        y4 = $\phi$(y2, y3) \\
        x3 = $\phi$(x1, x2) \\
        x4 = z4 - 1 \\
      \end{tabular}
  }}
  \putnode{b11}{b7}{20}{-20}{\symbox{11}{
      \renewcommand{\arraystretch}{.1}
      \begin{tabular}{@{}l@{}}
        z5 = $\phi$(z0, z6) \\
        y5 = $\phi$(y0, y7) \\
        x5 = $\phi$(x2, x6) \\
        y6 = y5 + x5 \\
      \end{tabular}
  }}
  \putnode{b12}{b10}{0}{-20}{\symbox{12}{
      \renewcommand{\arraystretch}{.1}
      \begin{tabular}{@{}l@{}}
        z6 = $\phi$(z4, z5) \\
        y7 = $\phi$(y4, y6) \\
        x6 = $\phi$(x4, x5) \\
      \end{tabular}
  }}
  \putnode{b15}{b12}{0}{-20}{\symbox{13}{
      \renewcommand{\arraystretch}{.1}
      \begin{tabular}{@{}l@{}}
        z8 = $\phi$(z7, z6) \\
        y9 = $\phi$(y8, y7) \\
        x8 = $\phi$(x7, x6) \\
        EXIT \\
      \end{tabular}
  }}

  \ncline{->}{b1}{b2}
  \ncline{->}{b2}{b3}
  \ncline{->}{b2}{b4}
  \ncline{->}{b3}{b5}
  \ncline{->}{b3}{b6}
  \ncline{->}{b4}{b7}
  \ncline{->}{b4}{b8}
  \ncline{->}{b5}{b9}
  \ncline{->}{b6}{b9}
  \ncline{->}{b6}{b10}
  \ncline{->}{b7}{b10}
  \ncline{->}{b8}{b11}
  \ncline{->}{b10}{b12}
  \ncline[nodesepA=-1]{->}{b11}{b12}
  \ncline{->}{b12}{b15}

  \nccurve[angleA=-155,angleB=135,nodesep=-.1,ncurv=2]{->}{b9}{b3}
  \nccurve[angleA=-35,angleB=45,nodesepB=-.2,ncurv=1.4]{->}{b12}{b8}
\end{pspicture}
\end{center}

}

%%\clearpage
%%%%%%%%%%%%%%%%%%%%%%%%%%%%%%%%%%%%%%%%%%%%%%%%%%%%
\item Consider the following flow graph in SSA form, with one instruction per block:

\begin{center}
\psset{xunit=1mm,yunit=.7mm}
\begin{pspicture}(0,0)(150,-230)
  \psframe(0,10)(150,-230)
  \putnode{b0}{origin}{75}{5}{\symbox{0}{ENTRY}}
  \putnode{b1}{b0}{0}{-20}{\symbox{1}{$a_1 = 3$}}
  \putnode{b2}{b1}{0}{-20}{\symbox{2}{$d_1 = 2$}}
  \putnode{b3}{b2}{0}{-20}{\symbox{3}{$d_3 = \phi(d_1, d_2)$}}
  \putnode{b4}{b3}{0}{-20}{\symbox{4}{$a_3 = \phi(a_1, a_2)$}}
  \putnode{b5}{b4}{0}{-20}{\symbox{5}{$f_1 =a_3 + d_3$ }}
  \putnode{b6}{b5}{0}{-20}{\symbox{6}{$g_1 = 5$}}
  \putnode{b7}{b6}{0}{-20}{\symbox{7}{$a_2 = g_1 - d_3$}}
  \putnode{b8}{b7}{0}{-20}{\symbox{8}{if $(f_1 <= g_1)$}}
  \putnode{b9}{b8}{-15}{-20}{\symbox{9}{$f_2 = g_1 + 1$}}
  \putnode{b10}{b8}{15}{-20}{\rsymbox{10}{if $(g_1 < a_2)$}}
  \putnode{b11}{b10}{-15}{-20}{\symbox{11}{$f_3 = \phi(f_1, f_2)$}}
  \putnode{b12}{b11}{0}{-20}{\symbox{12}{$d_2 = 2$}}
  \putnode{b13}{b10}{15}{-20}{\rsymbox{13}{EXIT}}

  \ncline{->}{b0}{b1}
  \ncline{->}{b1}{b2}
  \ncline{->}{b2}{b3}
  \ncline{->}{b3}{b4}
  \ncline{->}{b4}{b5}
  \ncline{->}{b5}{b6}
  \ncline{->}{b6}{b7}
  \ncline{->}{b7}{b8}
  \ncline{->}{b8}{b9}\Bput[0.0]{{\small T}}
  \ncline{->}{b8}{b10}\Aput[0.0]{{\small F}}
  \ncline{->}{b10}{b11}\Bput[0.0]{{\small T}}
  \ncline{->}{b9}{b11}
  \ncline{->}{b11}{b12}
  \ncline{->}{b10}{b13}\Aput[0.0]{{\small F}}

  \nccurve[angleA=215,angleB=155,nodesep=0,ncurv=1.2]{->}{b12}{b3}
\end{pspicture}
\end{center}

\begin{enumerate}
\item Draw the SSA edges (def-use chains) for the CFG. \hfill[5]
\item Perform Sparse conditional Constant Propagation for the CFG. Clearly show
\begin{enumerate}
\item The contents of the worklists (FWL and SWL) at each stage.
\item The SSA edge or the CFG edge being processed
\item The values that are updated during processing.
\end{enumerate}~\hfill[20]
\end{enumerate}
\end{enumerate}
\hrule
\[ T H E ~~ E N D \]
\end{document}

