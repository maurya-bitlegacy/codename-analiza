\documentclass[11pt]{article}
\usepackage{anysize}
\usepackage{amssymb}
\usepackage{amsmath}
\usepackage{paralist}
\usepackage{pstricks}
\usepackage{pst-text}
\usepackage{pst-node}
\usepackage{pst-rel-points}
\usepackage{multirow}
\usepackage{xspace}
\usepackage{wrapfig}
\usepackage{fancyvrb}

%% Page layout
\pagestyle{headings}
\markboth{}{Your Name and Roll Number:}
\begin{document}
\noindent \rule{\textwidth}{.2mm}
\begin{center}
{\large {\bf Quiz-2 CS618} }

Duration: 45 Minutes \hfill Max Marks: 50
\end{center} 
\noindent \rule{\textwidth}{.2mm}
\begin{itemize}
\item Write your name and roll number on the question paper and the
  answer book.
\item No explanations will be provided. In case of a doubt, make
  suitable assumptions and justify.
\end{itemize}
\noindent \rule{\textwidth}{.2mm}

\noindent Consider the  CFG below. All  variables are undefined to begin with (have {\rm undef} value in {\em entry}).
    \begin{enumerate}
    \item (10) Show the {\bf Dominator Tree} for the CFG. 
    \item (15) Show the {\bf Dominance Frontier} for every node. 
    \item (15) Show {\bf Iterated Dominance Frontier} for the nodes conaining the definitions of:
      \begin{center}
        \mbox{}\hfill(a)  variable x\hfill (b)  variable y\hfill (c) variable z\hfill\mbox{}
      \end{center}
    \item (10) Show the final CFG in SSA form.
    \end{enumerate}

\newcommand{\bb}[1]{{\small\bf B#1}}
\newcommand{\pp}[1]{{\phantom{\small\bf B#1}}}
  \psset{unit=1mm}
  \begin{pspicture}(0,0)(0,0)
    %\psframe(0,0)(65,80)
    \putnode{b1}{origin}{60}{-10}{\bb{1}
      \psframebox{\begin{tabular}{c}
                    {\em /*entry*\!/} \\
                    x = {\rm undef} \\
                    y = {\rm undef} \\
                    z = {\rm undef}
                  \end{tabular}
                }\pp{1}}
    \putnode{b2}{b1}{0}{-25}{\bb{2}\psframebox{z = 1}\pp{2}}
    \putnode{b3}{b2}{-15}{-15}{\bb{3}\psframebox{\begin{tabular}{c}
                                              x = 1\\
                                              z = 2
                                            \end{tabular}
                                          }\pp{3}}
    \putnode{b4}{b2}{15}{-15}{\bb{4}\psframebox{x = 2}\pp{4}}
    \putnode{b5}{b3}{0}{-15}{\bb{5}\psframebox{y = x + 1}\pp{5}}
    \putnode{b6}{b4}{0}{-25}{\bb{6}\psframebox{\begin{tabular}{c}
                                              z = x + 3\\
                                              x = 4 + z
                                            \end{tabular}
                                          }\pp{6}}
    \putnode{b7}{b6}{0}{-15}{\bb{7}\psframebox{z = x + 7}\pp{7}}
    \putnode{b8}{b7}{-15}{-15}{\bb{8}\psframebox{
        \begin{tabular}{c}
          {\rm print} x, y, z\\
          {\em /*exit*\!/}
        \end{tabular}}\pp{8}}
    \ncline{->}{b1}{b2}
    \ncline{->}{b2}{b3}
    \ncline{->}{b2}{b4}
    \ncline{->}{b3}{b5}
    \ncline{->}{b3}{b6}
    \ncline{->}{b4}{b5}
    \ncline{->}{b4}{b6}
    \ncloop[angleA=-90,angleB=90,loopsize=20,arm=5,linearc=.2,
    offset=1]{->}{b5}{b3}
    \ncline{->}{b5}{b8}
    \ncline{->}{b6}{b7}
    \ncline{->}{b7}{b8}
  \end{pspicture}
\end{document}
