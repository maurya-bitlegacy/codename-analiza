\documentclass[12pt]{article}
\usepackage{graphicx}
\usepackage{graphics}
\usepackage{anysize}
\usepackage{fancyhdr}
\usepackage{pstricks}
\usepackage{amssymb}
\usepackage{enumitem}

\setlist{nolistsep,leftmargin=*}
\setlength{\headheight}{15pt}
\pagestyle{fancyplain}
\lhead{\fancyplain{}{{\em Page \#\/}\thepage}}
\chead{CS618}
\rhead{\fancyplain{}{\textit{Roll No:}\rule{30mm}{0pt}}}
\lfoot{}
\cfoot{}
\rfoot{}

\newcommand{\tbd}[1]{{\red #1}}
\newcommand{\answer}[1]{{{\blue #1}}}
\newcommand{\ques}[1]{\bigskip\noindent{\bf Q#1.}}
\thispagestyle{empty}

\begin{document}

%{\Large\bf NAME: \hspace*{3in} ROLL NO:}

\begin{center}{\bf\Large
CS618: Program Analysis \\
Assignment 3 (Solutions)}
\end{center}

\ques{1} Let $(S, \wedge)$ be a semilattice. Let
  $f: S \rightarrow S$ be a function. Prove that the following
  two definitions for {\em monotonicity} of $f$ are equivalent:
  $$\forall x,y \in S: f(x \wedge y) \leq f(x) \wedge
  f(y)$$
  \begin{center} and \end{center}
  $$\forall x,y \in S: x\leq y \Rightarrow f(x) \leq f(y) $$

\answer{
  \begin{itemize}
  \item $\Rightarrow$: Let $x \leq y$. Then:\\
    $\begin{array}{rcl}
    x \wedge y = x &\Rightarrow& f(x \wedge y) = f(x) \\
    & \Rightarrow & f(x) \leq f(x) \wedge f(y) \\
    & \Rightarrow & f(x) \leq f(y) \qquad\qquad \mbox{// Recall properties of
      } \wedge
  \end{array}$
  \item $\Leftarrow$: \\
    $\begin{array}{rcl}
       x \wedge y \leq x &\Rightarrow& f(x \wedge y) \leq f(x) \\
       x \wedge y \leq y &\Rightarrow& f(x \wedge y) \leq f(y) \\
                         &\Rightarrow& f(x \wedge y) \mbox { is a lower bound on $f(x)$ and $f(y)$}\\
                         & \Rightarrow & f(x \wedge y) \leq f(x) \wedge f(y) 
  \end{array}$
  \end{itemize}
}

\bigskip
\ques{2} {\bf Knaster‐-Tarski Fixed Point Theorem}: Let
  $f : S\rightarrow S$ be a monotonic function on a complete
  lattice $(S,\vee,\wedge)$.  Using the concepts covered in
  class, prove that fix-points of $f$ (i.e. members of
  $\mbox{\rm fix}(f)$) form a complete lattice.

\answer{ 
    % 
    Let $Y$ be an arbitrary subset of $\mbox{\rm fix}(f)$.  We
    need to show that lub($Y$) and glb ($Y$) are in
    $\mbox{\rm fix}(f)$. We show lub$(Y) \in \mbox{\rm
      fix}(f)$. Ther other part is similar.

    Consider $y = \wedge Y$, and the set
    $X = \{ x \mid x \in S, x \leq y\}$.  
    \begin{center}
      \begin{tabular}[t]{rcp{.6\textwidth}}
        $(S,\vee,\wedge)$ is a complete lattice &$\Rightarrow$& $y \in S$  \dotfill(1) \\
        $y \leq y$ &$\Rightarrow$ & $y \in X$\dotfill(2) \\
        (1) and (2) &$\Rightarrow$ & $y \mbox{ is lub}(X,\vee,\wedge)$\dotfill(3)\\
        && $\mbox{glb}(S,\vee,\wedge)) \mbox{ is also glb}(X,\vee,\wedge)$\dotfill(4)\\
        && $X \subseteq S$\dotfill(5)\\
        (3), (4) and (5) &$\Rightarrow$ & $(X,\vee,\wedge)$ is a complete lattice\dotfill(6)\\

      \end{tabular}
    \end{center}

    Consider a restriction of $f$ over $X$. We show that $f: X \rightarrow X$. Consider $x \in X$ and $z \in Y$.
    \begin{center}
      \begin{tabular}[t]{rcp{.6\textwidth}}
        $y = \wedge Y$ &$\Rightarrow$& $x \leq y \leq z$\dotfill(7)\\
        (7) &$\Rightarrow$& $f(x) \leq f(z)$\dotfill(8)\\
        $z$ is a fixed-pt of $f$ & $\Rightarrow$ & $f(z) = z$\dotfill(9)\\
        (8) and (9) & $\Rightarrow$ & $f(x) \leq z$\dotfill(10)\\
        \multicolumn{3}{l}{Since $z$ is an arbitrary elememt in $Y$, and $y = \wedge Y$}\\
        &$\Rightarrow$ & $f(x) \leq y$\dotfill(11) \\
        $f(x) \in S$ and (11) &$\Rightarrow$ & $f(x) \in X$\dotfill(12)\\
        \multicolumn{3}{l}{Thus, $x \in X \Rightarrow f(x) \in X$. In other words, $f: X \rightarrow X$.}
      \end{tabular}
    \end{center}

    $(X,\vee,\wedge)$    is     a    complete     lattice    (6),
    $f:   X  \rightarrow   X$  is   a  monotonic   function  over
    $X$.  Therefore,  by   other  statements  of  Knaster-‐Tarski
    theorem, $f$ has a {\em greatest fixed-point} $z$ in $X$. But
    then, $z \in \mbox{\rm fix}(f), z \leq y = \wedge Y$ implying
    lub$(Y) = z \in \mbox{\rm fix}(f)$
        
}

\bigskip
\ques{3} An edge in a flow graph is a back edge if its head
  dominates its tail. Prove that every back edge is a retreating
  edge in every DFST of every flow graph.
\newcommand{\pentry}{\mbox{\rm ENTRY}}

  \answer{  Let $m  \rightarrow  n$  be a  back  edge. Then,  $n$
    dominates $m$.  Any path  from \pentry\ to  $m$ must  go through
    $n$. No  matter how  we construct DFST,  $n$ will  be touched
    before $m$.  Thus, DFS number  of $n$  will be less  than DFS
    number of  $m$ in  any DFST  $\Rightarrow$ $m  \rightarrow n$
    will be a retreating edge in any DFST.  }

\bigskip
\ques{4} In  a  flow  graph,  the  natural  loop  of  a  back  edge
  $a\rightarrow  b$  is $\{b\}$  plus the set  of nodes  that can
  reach $a$  without going through  $b$.  Prove that  two natural
  loops are either disjoint, identical, or nested.

\answer{
\newcommand{\pfree}[1]{\ensuremath{\stackrel{\overline{#1}}{\rightsquigarrow}}}
\newcommand{\phas}[1]{\ensuremath{\stackrel{{#1}}{\rightsquigarrow}}}

{\bf Notation:} $n_1 \pfree{n_2} n_3$ denotes a path from $n_1$ to $n_3$ that does not go through $n_2$. $n_1 \phas{n_2} n_3$ denotes a path from $n_1$ to $n_3$ that  goes through $n_2$ 


{\bf Lemma A:} If node $x$ is in the  natural  loop  of  a  back  edge $a\rightarrow b$, then $b$ dominates $x$.

{\bf Proof Hint:} Consider the path $\pentry\rightsquigarrow x\pfree{b}a$, and note that $b$ dominates $a$.


{\bf Proof of main statement}:   Assume to the contrary that natural loops for back edges $a_1\rightarrow b_1$ (say, $L_1$) and $a_2\rightarrow b_2$ (say, $L_2$) intersect, but are niether identical, nor nested. Thus, there is a node $x$ that belongs to both $L_1$ and $L_2$, a node $y_1$ that belongs to $L_1$ but not $L_2$, and a node $y_2$  that belongs to $L_2$ but not $L_1$.

Clearly both $b_1$ and $b_2$ dominate $x$ (From Lemma above). Thus, either $b_1$ dominates $b_2$ or $b_2$ dominates $b_1$ (Proof?). Without loss of generality, assume $b_1$ dominates $b_2$. Clearly, $b_2$ can not dominate $b_1$ so $b_1$ does not belong to $L_2$ implying that paths from $b_1$ to $a_2$, if any, must go through $b_2$.

Now consider the path $y_2\pfree{b_1} a_2 \rightarrow b_2 \pfree{b_1} x \pfree{b_1} a_1$: such a path must exist (why?), and does not involve $b_1$.  But then, $y_2$ belongs to $L_1$ - a contradiction.
}
%%%%%%%%%%%%%%%%%%%%%%%%%%%%%%%%%%%%%%%%%%%%%%%%%%%%
\end{document}

