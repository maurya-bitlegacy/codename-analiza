\mode<presentation>
{
  %\usetheme{umbc4}
  %\setbeamercovered{dynamic}
}
\usepackage[english]{babel}
% or whatever

\usepackage[latin1]{inputenc}
% or whatever

\usepackage{times}
\usepackage[T1]{fontenc}
\usepackage{etex}
% Or whatever. Note that the encoding and the font should match. If T1
% does not look nice, try deleting the line with the fontenc.
%--------------------------------------------------------------------------------------------
%--- Other packages
\usepackage{pstricks}
\usepackage{pst-node}
\usepackage{pst-rel-points}
\usepackage{bcprules}
\usepackage{graphicx}
\newcommand{\change}[1]{{\red #1}}
%%%%%% Declarations Defined in custom-foils
%\setfootline{\insertshortauthor, \insertshortinstitute \hfill   \insertshorttitle \hfill \insertframenumber/\inserttotalframenumber} 
\newcommand{\mypart}[1]{%
  \frame[plain]{\mbox{}\vfill \psshadowbox{\huge #1} \vfill\mbox{}}
}
\author[karkare]{Amey Karkare \\ \url{karkare@cse.iitk.ac.in}}
\date[]{\scalebox{0.3}{\includegraphics{iitklogo.epsi}}}%
\institute[CSE, IITK]{\url{http://www.cse.iitk.ac.in/~karkare/cs738}\\Department of CSE, IIT Kanpur}
\title[CS738]{CS738: Advanced Compiler Optimizations}
\subtitle[]{}


\subtitle[Types]{\ \\{\LARGE The Untyped Lambda Calculus}}

\begin{document}

\frame{\titlepage}

\frame{
  \frametitle{Reference Book}
  Types and Programming Languages by Benjamin C. Pierce
}
\newcommand{\tm}{\ensuremath{\mathsf t}}
\frame{
  \frametitle{The Abstract Syntax}
  \begin{tabular}{rcl@{\qquad}r}
    $\tm$ & := & $x$         &-- Variable\\ \pause
    & & $\mid \lambda x.\tm$ &-- Abstraction\\ \pause
    & & $\mid \tm\ \tm $     &-- Application \\ \pause
  \end{tabular}

  Parenthesis, (\ldots), can be used for grouping and scoping.
}

\frame{
  \frametitle{Conventions}
  \begin{itemize}[<+->]
  \item $\lambda x.\tm_1\tm_2\tm_3$ is an abbreviation for
    $\lambda x.(\tm_1\tm_2\tm_3)$, i.e., the scope of $x$ is as
    far to the right as possible until it is
    \begin{itemize}
    \item terminated by a {\bf )} whose matching {\bf (} occurs to the left of $\lambda$, OR
    \item terminated by the end of the term.
    \end{itemize}
  \item Applications associate to the left: $\tm_1\tm_2\tm_3$ to be read as $(\tm_1\tm_2)\tm_3$ and not as $\tm_1(\tm_2\tm_3)$
  \item $\lambda x y z.\tm$ is an abbreviation for  $\lambda x \lambda y \lambda z.\tm$ which in turn is abbreviation for $\lambda x.(\lambda y.(\lambda z.\tm))$.
  \end{itemize}
}

\frame{
  \frametitle{$\alpha$-renaming}
  \begin{itemize}[<+->]
  \item The name of a bound variable has no
    meaning except for its use to identify
    the bounding $\lambda$.
  \item Renaming a $\lambda$ variable, including all its
    bound occurrences, does not change the meaning of an expression.
    For example, $\lambda x. x\ x\ y$ is equivalent to
    $\lambda u. u\ u\ y$

    \begin{itemize}
    \item But it is not same as $\lambda x. x\ x\ w$
    \item Can not change free variables! 
    \end{itemize}
  \end{itemize}
}

\newcommand{\betared}{\ensuremath{\stackrel{\beta}{\longrightarrow}}}
\newcommand{\alpharen}{\ensuremath{\stackrel{\alpha}{\longrightarrow}}}

\frame{
  \frametitle{$\beta$-reduction (Execution Semantics)}
  \begin{itemize}[<+->]
  \item if an abstraction $\lambda x.\tm_1$ is applied to a
    term $\tm_2$ then the result of the
    application is
    \begin{itemize}
    \item the body of the abstraction $\tm_1$ with all free
      occurrences of the formal parameter $x$ replaced with $\tm_2$.
    \end{itemize}
  \item For example,
    \[(\lambda f \lambda x. f\ (f\ x))\ g \betared \lambda x. g\ (g\ x)\]
  \end{itemize}
}

\frame{
  \frametitle{Caution}
  \begin{itemize}[<+->]
  \item During $\beta$-reduction, make sure a free
    variable is not captured inadvertently.
  \item The following reduction is {\bf WRONG}
    \[ (\lambda x \lambda y. x) (\lambda x. y) \betared 
    \lambda y. \lambda x. y\]
  \item Use $\alpha$-renaming to avoid variable capture
    \[ (\lambda x \lambda y. x) (\lambda x. y) 
    \alpharen (\lambda u \lambda v. u) (\lambda x. y) 
    \betared 
    \lambda v. \lambda x. y\]
  \end{itemize}
}

\frame{
  \frametitle{Exercise}
  \begin{itemize}
  \item Apply $\beta$-reduction as far as possible
  \end{itemize}
  \begin{enumerate}
  \item $(\lambda x\ y\ z.\ x\ z\ (y\ z))\ (\lambda x\ y.\ x)\ (\lambda y. y)$
  \item $(\lambda x.\ x\ x) (\lambda x.\ x\ x)$
  \item $(\lambda x\ y\ z.\ x\ z\ (y\ z))\ (\lambda x\ y.\ x)\ ((\lambda x.\ x\ x) (\lambda x.\ x\ x))$
  \end{enumerate}
}

\frame{
  \frametitle{Church-Rosser Theorem}
  \begin{itemize}[<+->]
  \item Multiple ways to apply $\beta$-reduction
  \item Some may not terminate
  \item However, if two different reduction
    sequences terminate then they always
    terminate in the same term
    \begin{itemize}
    \item Also called the {\em Diamond Property}
    \end{itemize}
  \item Leftmost, outermost reduction will find
    the normal form if it exists
  \end{itemize}
}

\frame{
  \frametitle{Programming in $\lambda$ Calculus}
  \begin{itemize}[<+->]
  \item Where is the other stuff?
  \item Constants?
    \begin{itemize}
    \item Numbers
    \item Booleans
    \end{itemize}
  \item Complex Types?
    \begin{itemize}
    \item Lists
    \item Arrays
    \end{itemize}
  \item Don't we need data?
  \end{itemize}
  \pause\alert{Abstractions act as functions as well as data!}
}

\frame{
  \frametitle{Numbers: Church Numerals}
  \begin{itemize}[<+->]
  \item We need a ``Zero''
    \begin{itemize}
    \item ``Absence of item''
    \end{itemize}
  \item And something to count
    \begin{itemize}    
    \item ``Presence of item''
    \end{itemize}
  \item Intuition: Whiteboard and Marker
    \begin{itemize}
    \item Blank board represents Zero
    \item  Each mark by marker represents a count.
    \item  However, other pairs of objects will work
      as well
    \end{itemize}
  \item Lets translate this intuition into $\lambda$-expressions
  \end{itemize}
}

\frame{
  \frametitle{Numbers}
  \begin{itemize}[<+->]
  \item Zero = $\lambda m\ w.\ w$
    \begin{itemize}
    \item No mark on the whiteboard
    \end{itemize}
  \item One = $\lambda m\ w.\ m\ w$
    \begin{itemize}
    \item One mark on the whiteboard
    \end{itemize}
  \item Two = $\lambda m\ w.\ m\ (m\ w)$
  \item \ldots
    
  \item What about operations?
    \begin{itemize}
    \item add, multiply, subtract, divide, \ldots?
    \end{itemize}
  \end{itemize}
}

\frame{
  \frametitle{Operations on Numbers}
  \begin{itemize}[<+->]
  \item succ = $\lambda x\ m\ w.\ m\ (x\ m\ w)$
    \begin{itemize}
    \item Verify: succ N = N + 1
    \end{itemize}
  \item add = $\lambda x\ y\ m\ w.\ x\ m\ (y\ m\ w)$
    \begin{itemize}
    \item Verify: add M N = M + N
    \end{itemize}
  \item mult = $\lambda x\ y\ m\ w.\ x\ (y\ m)\ w$
    \begin{itemize}
    \item Verify: mult M N = M * N
    \end{itemize}
  \end{itemize}
}

\frame{
  \frametitle{More Operations}
  \begin{itemize}[<+->]
  \item pred = $\lambda x\ m\ w.\ x\ (\lambda g\ h.\ h\ (g\ m)) (\lambda u.\ w) (\lambda u.\ u)$
    \begin{itemize}
    \item Verify: pred N = N - 1
    \end{itemize}
  \item nminus = $\lambda x\ y.\ y\ \mbox{pred}\ x$
    \begin{itemize}
    \item Verify: nminus M N = max(0, M - N) -- natural subtraction
    \end{itemize}
  \end{itemize}
}
\frame{
  \frametitle{Church Booleans}
  \begin{itemize}[<+->]
  \item True and False
  \item Intuition: Selection of one out of two (complementary)
    choices
  \item True = $\lambda x\ y.\ x$
  \item False = $\lambda x\ y.\ y$
  \item Predicate:
    \begin{itemize}
    \item isZero = $\lambda x.\ x\ (\lambda u.\mbox{False})\ \mbox{True}$
    \end{itemize}
  \end{itemize}
}

\frame{
  \frametitle{Operations on Booleans}
  \begin{itemize}[<+->]
  \item Logical operations
    \begin{eqnarray*}
      and &=& \lambda p\ q.\ p\ q\ p\\
      or  &=& \lambda p\ q.\ p\ p\ q \\ 
      not &=& \lambda p\ t\ f. p\ f\ t
    \end{eqnarray*}
  \item The conditional operator $if$
    \begin{itemize}
    \item $if\ c\ e_t\ e_f$ reduces to $e_t$ if $c$ is True, and
      to $e_f$ if $c$ is False
    \end{itemize}
    \begin{eqnarray*}
      if  &=& \lambda c\ e_t\ e_f.\ (c\ e_t\ e_f)
    \end{eqnarray*}
  \end{itemize}
}

\frame {
  \frametitle{More\ldots}
  \begin{itemize}[<+->]
  \item More such types can be found at \\ \url{https://en.wikipedia.org/wiki/Church_encoding}
  \item It is fun to come up with your own definitions for
    constants and operations over different types 
  \item or to develop understanding for existing definitions.
  \end{itemize}
}

\frame{
  \frametitle{We are missing something!!}
  \begin{itemize}
  \item  The machinery described so far does
    not allow us to define Recursive
    functions
    \begin{itemize}
    \item Factorial, Fibonacci, \ldots
    \end{itemize}
  \item There is no concept of ``named'' functions
    \begin{itemize}
    \item So no way to refer to a function
      ``recursively''!
    \end{itemize}
  \item Fix-point computation comes to rescue
  \end{itemize}
}

\frame{
  \frametitle{Fix-point and $Y$-combinator}
  \begin{itemize}[<+->]
  \item A fix-point of a function $f$ is a value
    $p$ such that $f\ p = p$
  \item Assume existence of a magic
    expression, called $Y$-combinator, that
    when applied to a $\lambda$-expression, gives
    its fixed point
    \[Y\ f = f\ (Y\ f)\]
  \item $Y$-combinator gives us a way to apply
    a function recursively
    \end{itemize}
}

\newcommand{\factl}{\mbox{fact}}
\newcommand{\iszero}{\mbox{isZero}}
\newcommand{\mult}{\mbox{mult}}
\newcommand{\pred}{\mbox{pred}}
\newcommand{\one}{\mbox{One}}

\frame{
  \frametitle{Recursion Example: Factorial}
  \begin{eqnarray*}
    \factl &=& \lambda n.\ if\ (\iszero\ n)\ \one\ (\mult\ n\ (\factl\ 
              (\pred\ n)))\\
          &=& (\lambda f\ n.\ if\ (\iszero\ n)\ \one\ (\mult\ n\ (f\ (\pred\ n))))\ \factl \\ && \\ \pause
    \factl &=& g\ \factl
  \end{eqnarray*}
  \begin{itemize}[<+->]
  \item fact is a fixed point of the function
    \[ g = (\lambda f\ n.\ if\ (\iszero\ n) \one\ (\mult\ n\ (f\ (\pred\ n))))\]
  \item Using Y-combinator,
    \[\factl\ =\ Y\ g\]
  \end{itemize}
}

\frame{
  \frametitle{Factorial: Verify}
  \begin{eqnarray*}
    \factl\ 2 &=& (Y\ g)\ 2 \\\pause
           &=& g\ (Y\ g)\ 2 \quad\mbox{-- by definition of Y-combinator}\\\pause
           &=& (\lambda f n.\ if\ (\iszero\ n)\ 1\ (\mult\ n\ (f\ (\pred\ n))))\ (Y\ g)\ 2\\\pause
           &=& (\lambda n.\ if\ (\iszero\ n)\ 1\ (\mult\ n\ ((Y\ g)\ (\pred\ n))))\ 2\\\pause
           &=& if\ (\iszero\ 2)\ 1\ (\mult\ 2\ ((Y\ g) (\pred 2)))\\\pause
           &=& (\mult\ 2\ ((Y\ g)\ 1))\\\pause
           & & \ldots\\ 
           &=& (\mult\ 2\ (\mult\ 1\ (if\ (\iszero\ 0)\ 1\ (\ldots))))\\\pause
           &=& (\mult\ 2\ (\mult\ 1\ 1))\\\pause &=& 2
  \end{eqnarray*}
}

\frame{
  \frametitle{Recursion and $Y$-combinator}
  \begin{itemize}[<+->]
  \item Y-combinator allows to unroll the body
    of loop once---similar to one unfolding
    of recursive call
    
  \item Sequence of $Y$-combinator applications
    allow complete unfolding of recursive
    calls
  \end{itemize}
  \pause\alert{BUT, what about the existence of $Y$-combinator?}
}

\frame{
  \frametitle{$Y$-combinators}
  \begin{itemize}
  \item Many candidates exist
    \[Y_1 =  \lambda f.\ (\lambda x.\ f\ (x\ x))\  (\lambda x.\ f\ (x\ x)) \]\pause
    {\small\[Y = \lambda abcdefghijklmnopqstuvwxwzr.r(thisisafixedpointcombinator)\]}\pause
    \[ Y_{\rm funny} = TTTTT\ TTTTT\ TTTTT\ TTTTT\ TTTTT\ T \]\pause
  \item Verify that $(Y\ f)\ =\ f\ (Y\ f)$ for each
  \end{itemize}
}

\frame{
  \frametitle{Summary}
  \begin{itemize}[<+->]
  \item 
    A cursory look at $\lambda$-calculus
  \item Functions are data, and Data are
    functions!
  \item Not covered but important to know:  The power of
    $\lambda$ calculus is equivalent to that of Turing
    Machine (``Church Turing Thesis'')
  \end{itemize}
}

\end{document}
