\documentclass[11pt]{article}
\usepackage{anysize}
\usepackage{amssymb}
\usepackage{amsmath}
\usepackage{paralist}
\usepackage{pstricks}
\usepackage{pst-text}
\usepackage{pst-node}
\usepackage{pst-rel-points}
\usepackage{multirow}
\usepackage{xspace}
\usepackage{wrapfig}
\usepackage{fancyvrb}
\newcommand{\answer}[1]{}
%% Page layout
\pagestyle{headings}
\markboth{}{Your Name and Roll Number:}
\begin{document}
\noindent \rule{\textwidth}{.2mm}
\begin{center}
{\large {\bf Quiz-3 CS618} }

Duration: 45 Minutes \hfill Max Marks: 50
\end{center} 
\noindent \rule{\textwidth}{.2mm}
\begin{itemize}
\item Write your name and roll number on the question paper and the
  answer book.
\item No explanations will be provided. In case of a doubt, make
  suitable assumptions and justify.
\item There are 2 questions on 2 pages.
\end{itemize}
\noindent \rule{\textwidth}{.2mm}

\begin{enumerate}
\item ({\bf 15 Marks}) Perform {\bf Flow insensitive Subset-based} and {\bf
  Flow insensitive Equality-based} points-to analysis for
  each of the programs. Show only the {\em final} points-to
  information as points-to graphs.
\hrule
  \begin{verbatim}
(a)  if(...)
        p = &x;
     else
        p = &y;

      x = &a;
      y = &b;
     *p = &c;
     *y = &a;


(b)  if(...)
      y = &b;
     else
     *y = &a;

      p = &x;
      p = &y;
      x = &a;
     *p = &c;
  \end{verbatim}
\vfill\mbox{}
\begin{center} (P.T.O.) \end{center}
%\hrule
\clearpage
\item ({\bf 35 Marks}) In this problem, you have to perform the
  steps given below for {\em Inter-procedural Available
    Expressions} analysis for the expression {\tt a * b} for the
  following program. Assume m() is the main function.
  \newcounter{cline}\setcounter{cline}{0} \newcommand{\nline}{%
    \addtocounter{cline}{1} \thecline.\ } \renewcommand{\nline}{}
  
  \newcommand{\ind}{$\qquad$}
  \renewcommand{\arraystretch}{1.5}
  \begin{center}
    \begin{tabular}{@{}l@{\ind\ind}l@{\ind\ind}l}
      \begin{tabular}[t]{@{}l@{}}
        m() \{ \\ 
        \nline \ind call v() \\
        \nline \ind call u() \\
        \}
      \end{tabular}&
      \begin{tabular}[t]{@{}l@{}}
        u() \{ \\
        \nline   \ind a * b \\
        \ind if(\ldots) then \\
        \nline   \ind\ind call v() \\
        \ind else \\
        \nline    \ind\ind call u() \\
        \}
      \end{tabular}&
      \begin{tabular}[t]{@{}l@{}}
        v() \{ \\
        \ind if(\ldots) then \{\\
        \nline     \ind\ind b = 6 \\
        \nline     \ind\ind a * b \\
        \ind \}\\
        \}
      \end{tabular}
    \end{tabular}
  \end{center}
  Show the following steps to arrive at the answer:
  \begin{enumerate}
\item Draw the inter-procedural flow graph (any one of the two representations discussed in class). Clearly label the nodes, and distinguish the $E^0$ (intra-procedural) and $E^1$ (inter-procedural) edges.
\item For each procedure $p$, show the constraints for $\phi(r_p, n_p)$ for each node $n_p$ in the CFG of the procedure. Recall that $r_p$ is the entry node for procedure $p$. Assume the existence of flow functions $f_0, f_1, id$ etc for the underlying data flow framework.
\item Give the fixed-point solution for $\phi(r_p, n_p)$.
\item List the program points where the expression
  {\tt a * b} is available for the program.
\end{enumerate}
\hfill[Marks Distribution: 5+15+10+5]

\answer{ 
  \begin{center}
  \begin{tabular}{l@{\ind\ind}l@{\ind\ind}l}
    \begin{tabular}[t]{@{}l@{}}
      main() \{ \\
      \rnode{n00}{ ENTRY }\\
      \rnode{n01}{ $c_1:$ p() }\\
      \rnode{n21}{ \ ret p() }\\
      \rnode{n02}{ c * d }\\
      \rnode{n03}{ $c_2:$ q() }\\
      \rnode{n23}{ \ ret q() }\\
      \rnode{n04}{ EXIT }\\
      \}
    \end{tabular}&
    \begin{tabular}[t]{@{}l@{}}
      p() \{ \\
      \rnode{n05}{ \ind ENTRY }\\
      \rnode{n06}{ \ind a * b }\\
      \rnode{n07}{ $c_3:$ q() } \ind \rnode{n08}{    $c_4:$ q() }\\
      \rnode{n27}{ \ ret q()  } \ind \rnode{n28}{  \  ret   q() }\\
      \rnode{n09}{    \ind c = 5 }\\
      \rnode{n10}{ \ind EXIT }\\
      \}
    \end{tabular}&
    \begin{tabular}[t]{@{}l@{}}
      q() \{ \\
      \rnode{n11}{ \ind ENTRY }\\
      \rnode{n13}{ a = 5 } \ind \rnode{n15}{      b = 6 }\\
      \rnode{n16}{ \ind EXIT }\\
      \}
    \end{tabular}
  \end{tabular}

  \ncline{->}{n00}{n01}
  \ncline{->}{n21}{n02}
  \ncline{->}{n02}{n03}
  \ncline{->}{n23}{n04}

  \ncline{->}{n05}{n06}
  \ncline{->}{n06}{n07}
  \ncline{->}{n06}{n08}
  \ncline{->}{n27}{n09}
  \ncline{->}{n28}{n09}
  \ncline{->}{n09}{n10}

  \ncline{->}{n11}{n13}
  \ncline{->}{n11}{n15}
  \ncline{->}{n13}{n16}
  \ncline{->}{n15}{n16}

  \nccurve[linestyle=dashed,angleB=90]{->}{n01}{n05}
  \nccurve[linestyle=dashed,angleB=75]{->}{n03}{n11}
  \nccurve[linestyle=dashed,angleB=90]{->}{n07}{n11}
  \nccurve[linestyle=dashed,angleB=135]{->}{n08}{n11}
  \nccurve[linestyle=dashed,angleB=-135]{<-}{n21}{n10}
  \nccurve[linestyle=dashed,angleB=-45]{<-}{n23}{n16}
  \nccurve[linestyle=dashed,angleB=-90]{<-}{n27}{n16}
  \nccurve[linestyle=dashed,angleB=-135]{<-}{n28}{n16}

%  \ncline{->}{n10}{n21}
%  \nccurve[linestyle=dotted,angleA=-75]{->}{n16}{n23}
%  \nccurve[linestyle=dotted,angleA=-90]{->}{n16}{n27}
%  \nccurve[linestyle=dotted,angleA=-135]{->}{n16}{n28}

\end{center}
\begin{center}
  \begin{tabular}{l|l|l|l|l} \hline
    {\bf Pgm} & {\bf gen} & {\bf kill} & {\bf IN} & {\bf OUT}
    \\ \hline
    main() \{ & & & & \\ \hline
    \ind ENTRY & & & $\lambda, 00$& $\lambda, 00$\\ \hline
    \nline $c_1:$ call p() & & & $\lambda, 00$& $c_1, 00$\\ \hline
    \nline \ind c * d & c*d & & $\lambda, 00$& $\lambda, 01$\\ \hline
    \nline $c_2:$ call q() & & & $\lambda, 01$& $c_2, 01$\\ \hline
    \ind EXIT & & & $\lambda, 01$ & $-$\\ \hline
    \} & & & & \\ \hline & & & & \\ \hline
    p() \{ & & & & \\ \hline
    \ind ENTRY & & & $c_1, 00$&$c_1, 00$ \\ \hline
    \nline   \ind a * b & a*b & & $c_1, 00$&$c_1, 10$ \\ \hline
    \ind if(\ldots) then & & & $c_1, 10$&$c_1, 10$ \\ \hline
    \nline   $c_3:$ \ind call q() & & & $c_1, 10$&$c_1c_3, 10$ \\ \hline
    \ind else & & & $-$ & $-$ \\ \hline
    \nline    $c_4:$ \ind call q() & & & $c_1, 10$&$c_1c_4, 10$ \\ \hline
    \nline    \ind c = 5 & & c*d & $c_1, 00;\ c_1, 00$&$c_1, 00$ \\ \hline
    \ind EXIT & & & $c_1, 00$&$\lambda, 00$ \\ \hline
    \} & & & & \\ \hline & & & & \\ \hline
    q() \{ & & & & \\ \hline
    \ind ENTRY & & & $c_1c_3, 10$; $c_1c_4, 10$; $c_2, 01$ & $c_1c_3, 10$; $c_1c_4, 10$; $c_2, 01$ \\ \hline
    \ind if(\ldots) then & & & $c_1c_3, 10$; $c_1c_4, 10$; $c_2, 01$& $c_1c_3, 10$; $c_1c_4, 10$; $c_2, 01$ \\ \hline
    \nline     \ind\ind a = 5 & & a*b & $c_1c_3, 10$; $c_1c_4, 10$; $c_2, 01$& $c_1c_3, 00$; $c_1c_4, 00$; $c_2, 01$ \\ \hline
    \ind else & & & & \\ \hline
    \nline     \ind\ind b = 6 & & a*b & $c_1c_3, 10$; $c_1c_4, 10$; $c_2, 01$& $c_1c_3, 00$; $c_1c_4, 00$; $c_2, 01$ \\ \hline
    \ind EXIT & & & $c_1c_3, 00$; $c_1c_4, 00$; $c_2, 01$  & $c_1, 00$; $c_1, 00$; $\lambda, 01$  \\ \hline
    \} & & & & \\ \hline
  \end{tabular}

bit1 bit2: bit1 represents availability of a*b, bit2 of c*d
you can also initialize boundary with $\bot$ instead of 0. Meet at
each point, ignoring call string part will give the
availability.
 
\end{center}

}
\end{enumerate}
\end{document}

