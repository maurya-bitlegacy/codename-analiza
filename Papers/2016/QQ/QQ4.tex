\documentclass[11pt]{article}
\usepackage{anysize}
\usepackage{amssymb}
\usepackage{amsmath}
\usepackage{paralist}
\usepackage{pstricks}
\usepackage{pst-text}
\usepackage{pst-node}
\usepackage{pst-rel-points}
\usepackage{multirow}
\usepackage{xspace}
\usepackage{wrapfig}
\usepackage{fancyvrb}
\usepackage{bcprules}
\usepackage{dashrule}
\newcommand{\answer}[1]{}
%% Page layout
\pagestyle{headings}
\markboth{}{Your Name and Roll Number:}
\begin{document}
\noindent \rule{\textwidth}{.2mm}
\begin{center}
{\large {\bf Quiz-4 CS618} }

Duration: 45 Minutes \hfill Max Marks: 50
\end{center} 
\noindent \rule{\textwidth}{.2mm}
\begin{itemize}
\item Write your name and roll number on the question paper and the
  answer book.
\item No explanations will be provided. In case of a doubt, make
  suitable assumptions and justify.
\item There are 2 questions on 2 pages.
\end{itemize}
\noindent \rule{\textwidth}{.2mm}

\newcommand{\tm}{\ensuremath{\mathsf t}}
\newcommand{\ty}{\ensuremath{T}}
\newcommand{\val}{\ensuremath{v}}
\begin{center}{\bf Useful Definitions}
\end{center}
  \begin{itemize}
  \item Simply Typed $\lambda$-terms
    \begin{center}
    \begin{tabular}{rcl@{\qquad}r}
      $\tm$ & := & $x$         & {\em -- Variable}\\ 
            & $\mid$ & $\lambda x:\ty.\ \tm$ &{\em -- Abstraction}\\
            & $\mid$& $\tm\ \tm $     &{\em -- Application} \\
    \end{tabular}
    \end{center}
  \item The Set of Values
    \begin{center}
      \begin{tabular}{rcl@{\qquad}l}
        \val\ &:=& $\lambda x:\ty.\ \tm$ &{\em -- Abstraction Value}
      \end{tabular}
    \end{center}
  \item The Evaluation Rules
    \infrule[E-App1]{\tm_1 \rightarrow \tm'_1}{\tm_1\ \tm_2 \rightarrow \tm_1'\ \tm_2} 

    \infrule[E-App2]{\tm_2 \rightarrow \tm'_2}{\val\ \tm_2 \rightarrow \val\ \tm'_2} 

    \infax[E-AppAbs]{(\lambda x:\ty_1.\ \tm_1)\val_2 \rightarrow [x\mapsto \val_2]\tm_1}
  \item   The Typing Rules
  \infrule[T-Abs]{\Gamma,x:\ty_1 \vdash \tm_2:\ty_2}{\Gamma\vdash\lambda x:\ty_1.\ \tm_2 : \ty_1\rightarrow \ty_2} 

  \infrule[T-Var]{x:\ty \in \Gamma}{\Gamma\vdash x:\ty} 

  \infrule[T-App]{\Gamma\vdash \tm_1:\ty_1\rightarrow\ty_2\andalso\Gamma\vdash \tm_2:\ty_1}{\Gamma\vdash \tm_1\ \tm_2:\ty_2}

  \end{itemize}
\hfill
\begin{center} (P.T.O.) \end{center}
\clearpage
\begin{enumerate}
\item ({\bf 20 Marks}) For each of the term $\tm$ below, find the
  types $\ty_1, \ty_2,$ etc. such that \tm\ has a valid type
  \ty. If such a type can not be found, show why.\footnote{You
    might want to use $\alpha$-renaming to avoid issues with
    variable name reuse.}

    \begin{itemize}
    \item $\lambda x:\ty_1\ y:\ty_2.\ x\ y$
    \item $\lambda x:\ty_1\ y:\ty_2.\ x\ y\ y$
    \item ($\lambda x:\ty_1\ y:\ty_2.\ x\ y\ y$) ($\lambda x:\ty_3\ y:\ty_4.\ x\ y$)
    \end{itemize}
\newcommand{\comp}{\mbox{complete}}
  \item ({\bf 30 Marks}) We define a $\lambda$-calculus program
    to be {\bf COMPLETE} if it does not contain any free
    variables (variables that are not bound by any $\lambda$ in
    the program). 
    \begin{enumerate}
    \item\ [10] Complete the following analysis that checks for
    complete $\lambda$-calculus programs.
    \infrule[Comp-Var]{x:\ty \in \Gamma}{\Gamma\vdash \comp(x)}\ \\[3pt]

    \infrule[Comp-Abs]{\hdashrule{12cm}{1pt}{1pt}}{\Gamma\vdash\comp(\lambda x:\ty_1.\ \tm_2)} \ \\[3pt]

  \infrule[Comp-App]{\hdashrule{12cm}{1pt}{1pt}}{\Gamma\vdash \comp(\tm_1\ \tm_2)}\ \\[3pt]
\item\ [10 + 10] Define {\em Progress} and {\em Preservation} specific for the  analysis rules above. Explain the intuition behind the definitions.
  \end{enumerate}

\end{enumerate}
\end{document}

